\chapter{What Is a Good Collective Decision?}
\label {apx:on-knowledge}

A group must arrive not just at a result, but at a correct result to make a good collective decision.  This is not necessarily the \textit{best} result, but one which is an improvement on the status quo, or is the best over all alternatives presented.

To do this, the collectivity requires knowledge; but what does this mean?  The individuals must have knowledge pertaining to that about which they make decisions, but that does not mean the collectivity has this knowledge, nor that all members must have this knowledge.  Further, what does it mean to ``know'' something?

\section{Knowing}

We can describe ``Knowing'' as the situation when a person $\mathit{P}$ knows a thing $\mathit{T}$ \autocites[57]{Tideman2006}{Nozick1983}, such that:

\begin{enumerate}
    \item $\mathit{T}$ is true.
    \item $\mathit{P}$ believes $\mathit{T}$ is true.
    \item $\mathit{P}$ has good reason to believe $\mathit{T}$ is true.
\end{enumerate}

A person can't very well ``know'' something that is not true; they can only be wrong.  Likewise, someone who doesn't believe something true is ignorant, or just wrong.  Finally, even if the broken clock is right twice a day, it's right for some bizarre reason that has nothing to do with reality:  if you think trees sequester carbon because space aliens told you so, you're not wrong, but you're clearly delusional and don't actually know anything about trees.

The biggest problem is we can never actually prove $\mathit{T}$ is true.  The \textit{second} problem is we all have lives and need to work from knowledge others have given us, which may all sound good yet have holes we can't recognize due to our ignorance.  Our education system, the general validity of knowledge that holds up to scrutiny, and the real world doing what we'd expect gives us a reason to believe all these things are true, so we're generally close enough.

There are \textit{serious} limitations with being ``close enough.''  Until recently in history, almost everyone understood that one human being could obviously own another human being, backed up by history, religion, and good common sense.  Common sense, being the culmination of one's own biases and prejudices, shows exactly the problem:  it is obvious that arming teachers protects against school shootings, because the teachers can shoot back; it is also obvious that arming the teachers makes them a target for school shooters as both a threat and way to get more guns and ammo.  Which one you think you ``know'' depends less on which is true and more on what you believe based on your entire experience in life.

\begin{todo}
    Rewrite this.  Too much of a rant, although raises a few interesting points as-is.
\end{todo}
Today, almost nobody thinks human beings can be property; but do people really ``know'' that?  Condorcet makes many claims on the subject, including a thought experiment where a person sells himself to another \autocite{Condorcet1781}—an absurdity he had to set aside, since a person being property would immediately \textit{forfeit the payment}.  He still showed the entire concept defied natural law.

People today generally only believe slavery is wrong because they were told so, while those politically active generally believe slavery is wrong because of—that is, in the context of—normative ideals of civil rights.  Slavery cannot be defended against the simple arguments of democratic legitimacy, where the slave does not have rights, but is nonetheless a self-aware being able to reason on morality and thus possesses the only qualities which give \textit{anyone} any rights at all \autocite{Condorcet1789}.

One might decide that ``slaves'' and ``slavery'' never existed and cannot exist:  there is only the highest crime of tyranny perpetrated against the oppressed, and all law contrary to this fact is illegitimate, the mere existence of which is a violent crime against those falsely claimed to be property and all who could possibly be falsely claimed to be property.  By this view, Condorcet often claimed society could not bear the burden of compensating the emancipated, but rather they were owed this by those purported to have ``owned'' them, and that taking property from those to give to the emancipated was not unjust because their claim to property gained by such unjust acts was illegitimate.

Whether or not you believe any of that, it's clear that these are either true or not, and that someone may have good reason to believe these are true, and so they may know these things \textit{if} they are actually true.

\section{What You Know}

In the strictest sense, it is impossible to ``know'' anything; but this is silly.  What we ``know'' can be explained by Bayes's Theorem, which we need not understand to appreciate.  As explained by Tideman \autocite{Tideman2006}, given two possible states of the world $A$ and $B$ known to have probabilities $P_A$ and $P_B$ before the incorporation of new data $D$, and the probabilities of observing $D$ given either is $P_D|A$ and $P_D|B$, then observing $D$ produces two probable states:

\centerline{
$\frac{P_A\cdot P_D|A}{P_A\cdot P_D|A + P_B\cdot P_D|B}$ and $\frac{P_B\cdot P_D|B}{P_A\cdot P_D|A + P_B\cdot P_D|B}$
}

All that noise just says the likelihood of $A$ or $B$, given everything you know now, will change if you learn $D$; how it changes is dependent on whether $D$ supports $A$ or $B$ and how much impact it has on the likelihood of either being correct.

Consider elections:  the more people vote, the more democratic an election is.  The best outcome is achieved by voting.  This is supported by common sense, and by the obvious observation that a losing candidate needs more votes to win.

Add to that the concept of spoilers.  The winner may not have a majority, and the presence of a loser may change the winner.  You might believe that until you come onto the knowledge that voters who vote for the spoiler would have not voted at all, in which case the outcome doesn't change; and so you weigh how likely they would be to vote for one or the other, based on what you know about the spoiler candidate and their similarity to the other two.  You don't \textit{really} know how the voters would have voted, but it's more reasonable that a more-liberal voter would instead vote for the more moderate liberal than the conservative if the most liberal candidate vanished, and absurd that they would vote for the conservative.

This brings us to the superiority of runoffs, which eliminate the spoiler candidate.  You know the spoiler only has a minority of votes and will go away, and their votes will transfer to the candidate who has a majority over the other.  This is correct; however, the spoiler might have a majority over \textit{both} candidates, individually, making them the Condorcet winner.  That means between two candidates, the addition of the third can \textit{eliminate} this winner and elect the original loser.  This comes with the knowledge that the spoiler can be the \textit{second} choice of voters who vote for the original loser.

You know, however, that the winner has \textit{a} majority—although now you realize there are multiple majorities and every candidate except perhaps one has \textit{a} majority—and so this still has merit…until you are either told or reason yourself (synthesize the data $D$ from all your existing data) that we can always add another losing candidate, so in a sense every candidate \textit{always} has a majority over someone who could have run but didn't.

The more people vote, the more democratic the result, correct?

Under Instant Runoff Voting, if fewer voters come out to vote for this losing candidate, then the candidate may be eliminated first.  That causes their second choice—rather than their \textit{last} choice—to win, meaning they're better off if fewer of them vote.  A better strategy is to vote for their second favorite instead, so voting is better than not voting; but voting for the candidate you want to \textit{win} can't cause your second favorite to \textit{lose} under plurality if they would have \textit{won} had you just not voted at all, except in the case where the candidate for which you vote wins.

Instant Runoff Voting also fails monotonicity, meaning putting your least-favorite candidate first on your ballot can cause your favorite candidate to win.  This is rare, and nonsense in practice, but possible.  Further, a voting system where every vote is equal \textit{must} have this property, but it may be utterly rare and prone to occur only in the most absurd contrivances.

Each of these works from faulty or incomplete knowledge.  It is presumed by many that those who vote for a spoiler ``take votes away from the real winner,'' due to an implicit assumption that those voters would have instead voted for the losing candidate.  Runoffs are presumed to combat this, but they can make it worse.  The third party candidate is assumed to not have majority support, but this changes when realizing they may be literally every voter's second choice, and so have a majority over every candidate—called the Condorcet winner.

Here's one more:  someone who gets no votes might be the candidate the voters want to elect.

Consider a body of voters, all of whom are candidates in an election.  Every voter votes themselves first, except for candidate $C$ who realizes candidate $A$ is actually better.  Candidate $C$ is \textit{literally everyone's second choice}.  That means candidate $A$ has the most votes, while candidate $C$ has an almost unanimous victory over each and every other candidate.

That means if you select any candidate as the winner and then have a vote between them and $C$, they will lose with every vote against them except the one they cast for themselves, and the one $C$ cast for $A$.

Under plurality, $C$ receives zero votes.

Bayes's Theorem is a bunch of confusing math that says that from everything you know, it's highly-likely that a candidate who gets zero votes should absolutely \textit{never} be elected; but this new data $D$ makes it suddenly highly likely that circumstances exist—unlikely to occur, but yet are now theoretically possible—wherein the ``will of the people,'' the actual correct person to elect, the most democratic winner, is the one who literally nobody ``voted for.''  Further, it says that the likelihood of you eventually discovering you are wrong is the likelihood that $D$ appears and causes this situation where what you decided—for good reason!—must have been true is actually false.

We ``know'' what we have good reason to believe.  We may be wrong, nonetheless.  Language is imprecise, and all of this above explains what we are saying when we claim we ``know'' something, even if we don't at the time believe (or care) about this ridiculously technical implication of knowledge.