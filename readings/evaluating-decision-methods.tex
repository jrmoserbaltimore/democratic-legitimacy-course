\chapter{Evaluating Collective Decision Procedures}
\label{apx:evaluating-dm}

Tideman provides a taxonomy of criteria for evaluating collective decision procedures \autocite{Tideman2006}, shown in \prettyref{fig:decision-evaluation-criteria}.  This document describes these in brief.



\begin{figure}[ht]
    \centering
    \begin{tikzpicture}[
        %sibling distance=10em,
        font=\small,
        edge from parent path={(\tikzchildnode.north) -- ++(0,0.5cm)
            -| (\tikzparentnode.south)},
        level distance=1.75cm,
        every node/.style = {
            shape=rectangle, rounded corners,
            draw=blue!75,
            align=center,
            top color=white,
            bottom color=blue!20,
            drop shadow}
        ]

        \draw (0,0) [sibling distance=15em] node {Evaluation Criteria}
        child { node {Efficiency}
            [sibling distance=5em] child { node {Outcome\\Efficiency} }
            child { node {Procedural\\Efficiency} }
        }
        child { node {Equity}
            [sibling distance=6em] child { node {Symbolic\\Equality} }
            child { node {Material\\Equality} }
            child { node {Productivity-\\Based\\Equality} }
        }
        child { node {Stability}
            [sibling distance=6em] child { node {Restoration} }
            child { node {Preservation} }
        };
    \end{tikzpicture}
    \caption{\label{fig:decision-evaluation-criteria}A taxonomy of criteria for evaluating collective decision procedures \autocite[35]{Tideman2006}.}
\end{figure}

\section{Efficiency}

Efficiency deals with the degree of achievement and at what cost.  These are the same considerations seen in $k$-majority rule, where a higher $k$ value increases outcome efficiency at the cost of lower procedural efficiency.

\begin{definition}{Procedural Efficiency}
    ``The magnitude of resources and psycho-social costs of magking decisions''  \autocite[36]{Tideman2006}.
\end{definition}

Procedural efficiency includes, primarily, the time required to reach a decision.  A higher $k$-majority, such as $k=\nicefrac{2}{3}$, will require more negotiation and more time to reach a decision.  There are other costs, such as the deliberate destruction of resources seen in extortion.  Extortion can occur in any setting, for example under simple majority rules when a few hold-outs refuse to pass a law unless they receive special favors.  The law may alleviate some ongoing crisis, and so this delay causes economic or even bodily harm.

\begin{definition}{Outcome Efficiency}
    The total utility gained by the outcome of a decision, relative to some specific alternative.
\end{definition}

Economists generally measure these outcomes in terms of money.  Buchanan and Tullock compare the outcome of a decision to the current state, or to anarchy, to determine the cost and create a philosophical basis for constitutional consent.

Tideman observes that the maximization of outcome and procedural efficiency—of total efficiency—is optimal only if individuals are completely selfish and if money has the same value to all persons, or cannot by any rule be predicted to have more or less value to any person.  These assumptions are unreasonable, so the pure maximization of both outcome and procedural efficiency is not necessarily in the greatest interest of the collectivity.  In legislatures, a higher $k$-majority may increase outcome efficiency more than procedural efficiency, yet may be desirable because of a greater value placed on better and more broadly representative legislation.

\section{Equity}

Equity is a core criteria for evaluating a decision making process:  people generally prefer processes producing fair outcomes.  Tideman's basic definition of equity is the same as distributive justice.

\begin{definition}{Equity}
   A distribution of goods that conforms to the shared understandings regarding what people ought to receive.
\end{definition}

Tideman further describes equity as symbolic, material, and productivity-based equality.  Each of these has particular implications for collective decisions.

\begin{definition}{Symbolic Equality}
    Equality in basic rights, treatment, or responsibilities.
\end{definition}

Symbolic equality is exemplified in ideals such as each person having only one vote, rather than special privilege for the aristocracy; of the application of law equally to all persons; and of equal opportunity, such as by non-discriminatory housing and hiring practices.

Many find symbolic equality an important ideal, although it in no way guarantees anything material.  For example:  one-person-one-vote may be expressed by each person casting one ballot in a plurality election, which may disenfranchise a great many voters or even every majority when there are several candidates.  Equal application of the law need not take into account the socioeconomic conditions leading a person to crime, even if caused by ineffective public policy, and besides does not require the law to be sensible or fair.  Non-discriminatory hiring practices don't require hiring those who have had unequal educational opportunities and so are unqualified for the job.

We may consider these problems in terms of material equality.  Rather than one-person-one-vote, an electoral system must be designed to make all votes equal.  Public policy may consider crime in its socioeconomic context and resolve the underlying problems for each individual, rather than applying crude punitive justice.  Equal access to education must provide not just a school and a curriculum, but an effective educational system preparing students for professional and civic life so that they may have equal opportunity to jobs.

\begin{definition}{Material Equality}
    Equality in material outcomes and in access to resources.
\end{definition}

Matters of material equality include social insurances such as universal healthcare, universal family care, and public retirement and disability pensions.  Minimum wage laws provide material equality for those at the lowest income levels, ensuring equal work receives some reasonable amount of pay compared to the average.  Progressive income taxes provide material equality because additional resources have more value for those with little than for those with a great abundance of wealth.  In effect, material equity accounts for the needs and opportunities of the individual and in physically addressing disparities as such.

Tideman defines productivity-based equality as the third form of equity.

\begin{definition}{Productivit-based Equality}
    Allocation of resources according to the value each person produces.
\end{definition}

Productivity-based equality shows up in debates over minimum wage, in which people claim low-paid workers simply don't provide value and are unproductive.  Those who do a great deal of work but produce little are compensated little; those who do little work but produce a great deal of value are compensated greatly.  This is by far the most debatable form of equity in a technical and philosophical sense:  if all the low-productivity workers went away, there would be no food at the grocery store, among other things, and society would collapse.  Likewise, the economic ``value'' of a worker's output is measured in its price, which is the combination of wages and profits taken along its entire cycle of production; while goods with the same value as such may have more or less value to different people or even in general.  The concept immediately fails any exercise in critical thinking, yet also is obviously based on reasonable premise, as different work does produce different physical amounts of goods and is materially more or less productive.

\section{Stability}

