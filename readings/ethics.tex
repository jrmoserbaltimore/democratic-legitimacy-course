
\label {apx:ethics}
\chapter{Ethics in Public Policy}

\nocite{Tideman2006}

Morality, ethics, and justice are core to civilized society, and philosophers have debated these for centuries.  In philosophical terms, the three often share the same meaning; whereas in every day use, they have overlapping or different meaning.

One possible framework is to consider morality, ethics, and justice as separate concepts.

We may define morality as individual, personal beliefs.  There is no real reason why a certain thing may be immoral, for example the use of intoxicating drugs like alcohol has been acceptable in nearly all societies while the use of cannibis has long been considered so immoral as to be made illegal, despite alcohol having arguably greater negative consequences for those who use it to excess and those around them.

\begin{definition}{Morality}
    Individual, personal beliefs about what is right or wrong, without a substantial reason.
\end{definition}

The term ``ethics'' appears in codes of ethics, ethics committees, ethics rules, and other formalized contexts.  We might define ethics, then, as a set of rules, whether they be prescribed actions such as that lawyers are to keep confidential all discussions with their clients, or frameworks by which to evaluate a situation and determine if it is ethical.

\begin{definition}{Ethics}
    Formalized rules describing what actions are ethical, or by what method an action may be evaluated to determine if it is ethical.
\end{definition}

Justice has always meant fairness in a great many theories, and people in general seem to believe what they perceive as fair is just.  In modern social contexts, the term is used to describe the fairness of a society, in reference to the treatment of people of various races by the criminal justice system, or to public policy regarding socioeconomic conditions which deny people the opportunity to advance in society regardless of their own efforts.

\begin{definition}{Justice}
    A condition by which the treatment of the members of society is considered fair.
\end{definition}

Theories of justice encompass a much greater scope than this definition, and are relevant to theories of ethics in public policy.  Justice spans the debate over what intervention society owes individuals in poverty or other conditions, what protections the law must provide against discrimination or other unjust treatment of one person by another, and even the foundations of criminal justice.  Beyond ethics as described here, however, other more diverse philosophical definitions of justice exist.

\section{The Good of the Whole}

Utilitarianism dominated the ethical frameworks of political economy in the ninteenth century.  The ``right thing to do'' was what provided the greatest good for the greatest number of people, consistent with the utilitarian theory of justice espoused by Stuart Mill.  This ``good of the whole'' remains attractive and provides a great deal of political rhetoric.  It forms the foundations of enlightenment-era ideals, modern progressive ideals, and populist rhetoric.

In the case of populism, an individual class is made out as the cause of all of society's ills, and policy to serve society is valued more for the degree of harm it does to this class than the degree of good; this is subtle, and identifiable in that many statements focus on a certain class, and policy described as good for the masses is presented with heavy emphasis on claims that this class is not only preventing the policy from being put into place, but cannot fundamentally exist if the policy is in place.

Historically, right-wing populism has focused on race and national identity, blaming the plight of the masses on Jews, blacks, immigrant foreigners, or meddling organizations such as foreign churches.  Left-wing populism has focused on elites, including career politicians, wealthy individuals, and large businesses; small businesses and labor unions are placed as the liberation of the worker, so long as their business remains small.  Regardless, it carries an underlying theme:  society's ills must be for the benefit of a certain class, and correcting society's ills must necessarily be at the expense of this class, and that this class is in general evil and deserves to face this expense, at once defining them as the Child of Omelas and the tyrant.

The argument that a class must face an expense and deserves to face that expense is often, but not always, a simple rhetorical tool.  Most societies universally agree slavery is unjust; and the Marquis de Condorcet wrote that emancipation cannot possibly deprive the slavers of property, despite such a policy coming at their direct expense, because they have stolen the lives of these human beings whom shall be freed \autocite{Condorcet1781}.

\section{Pareto and Kaldor-Hicks}

Vilfredo Pareto first wrote that a certain policy change is beneficial if it improves the welfare of some and does not decrease the welfare of any; whereas if it is good for some but at the expense of others, then we cannot claim it is good for the whole.  This change is known as a ``Pareto improvement,'' and we may say that a policy satisfying this is Pareto or Pareto-efficient.

The Pareto principle has several critical flaws.  First, it does not consult those in the collectivity:  if all involved agree that the new way is better, even if some will be less well off, the change is still not ethical under this strict framework.  Second, it is nearly impossible to satisfy—not just in basic principle, but also in that no small changes may satisfy Pareto, but there may be a Pareto improvement by moving to a fundamentally different policy in a way that almost never occurs and may be infeasible.

John Hicks and Nicolas Kaldor had produced a framework in which if a change in policy increases welfare such that the winners can fully compensate the losers, then the policy is ethical—even if no such compensation occurs.  This is a ``potential Pareto improvement,'' as carrying out the compensation makes the outcome Pareto-efficient.

\section{The Income Effect}

Pareto suggested we cannot compare the utility between two people.  This is easy to explain:  as a basic economic principle, the marginal utility of any additional income decreases as income increases.  What something is worth then becomes a function of what someone has.

In a practical sense, the first dollars a person receives, being poor, will go toward necessities:  food, shelter, clothing, safety.  Further income goes to luxuries, such as a larger house, better food, television, video games, and the like.  While a person may be able to afford all of their basic needs and a brand-new and expensive luxury car, they will settle for a good-quality, more-basic used car so they can spend the remaining money on other things; but when a person has a higher income, they buy the luxury car instead.

This change from a low-cost vehicle to a luxury vehicle occurs because the car is at the margin.  A car may come first, and be seen as a nacessity; but when it comes to ``adding'' luxury to the car, video games and cable TV come first.  Thus a person with less income will buy the car and not the Nintendo Switch; and with a little more income, they will buy the car and the Nintendo Switch; and with more income than that, they will buy the Nintendo Swith and a \textit{luxury car}, the added luxury being ``at the margin'' i.e. valued less than the video games.

The only sensible interpretation of this is that the first dollars are worth more than the last.  If a person has \$1,000 and buys a certain basket of goods, and if instead having \$2,000 buys this basket plus a second basket of different goods, then it is plain that they would not trade the first basket away to receive the second.  Despite this, each basket is purchased for \$1,000.  The first basket is worth more to the person—has higher value—than the second, and so the first \$1,000 of income has higher value than the next \$1,000 of income.

This creates a large degree of difficulty in policy ethics.

If a policy increases taxes on top income earners and redistributes that money to low-income earners, that policy is an improvement in total welfare; however, it is not a Kaldor-Hicks improvement:  the money only has higher value for the winners, and so they cannot possibly compensate the losers, therefor it is impossible to make this transfer a Pareto improvement.  Any form of taxation—and thus any government policy—will run into these complexities.

\section{Ethics and Consent}

Tibor Scitovsky noticed that when changing between policies A and B, a movement from Policy A to Policy B may be an improvement under Kaldor-Hicks, yet at the same time a movement from policy B to policy A may also be an improvement.  To solve this, he gave a double criterion, an approximation of which is as such:  a change is an improvement in social welfare if those benefitting can persuade those bearing the cost to accept the change, but those bearing the cost are simultaneously not able to persuade those benefitting to remain in the original situation.  The theorem in full is more complex than this, but at its core it highlights an important prospect:  a person can consent to bearing a cost to provide a benefit to another.

More recently, John Rawls, James Buchanan, and Gordon Tullock describe frameworks of consent to the rules of a society.  The Rawls-Buchanan-Tullock framework assumes unanimous consent if a person would be expected to agree to a particular rule change in a state where they do not know who they are in society.  Without knowing if you are a wealthy business owner or a low-paid worker, and with the assumption that a higher minimum wage is a cost to the business owner—which is intuitive, but not necessarily true—you may weigh the options and decide on a higher minimum wage \textit{just in case} you discover afterwards that you are a low-wage worker, but on a reasonable minimum wage nonetheless.  With that assumption, all have agreed to some reasonable minimum wage policy, if we believe the presented policy would be accepted under these conditions.

These conditions of consent give a way to confront the income effect and its implications for policy ethics.  Policy should, where possible, provide just compensation, and if not should be a Kaldor-Hicks improvement, and if otherwise should at least be reasonable when considered in the context of the Rawls-Buchanan-Tullock framework.
