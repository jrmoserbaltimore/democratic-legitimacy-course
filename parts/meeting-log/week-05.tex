% vim: sw=4 ts=4 et

\section{Week 5}

At this point, have added several annotated bibliography entries.  Stalled, and spoke with Dr. Kassner.

Stalled, want to know where to go from here with the research I've pulled together.  Largely, the democratic legitimacy stuff is extremely interesting, but I'm uncertain how to integrate it to the greatest degree I'd like.

Dr. Kassner gave advice on building a course, which gives me a direction and a framework to use.

A good course uses a ``narrative arc'' to lead the class through the material and the reasoning involved.  It's not just a walkthrough of content, but through thinking, leading the students to ask questions and make their own considerations.

This reminds me of McGrath opening his American Political Thought course by explaining that he has certain beliefs about the political ideals and the correct manner of running the nation, and that he hopes to convince us by the end of the course of the same.  It struck me as very direct and open, a sort of protection that allows students to critically evaluate the information they're given while freeing the professor to make normative claims.

Dr. Kassner suggested opening with some ideals, then showing flaws, a structure I'd heavily considered for the first class in particular, but not the whole course.  My first lesson simply makes a normative assertion of what democracy is, of what we assume it means; and then starts asking strange questions like ``what is a majority?'' and destroying every obvious answer the students might conceive.  I intended this to bait the class into making the expected declarations, and then quickly dismantling everything they thought they knew.  This makes it quite clear \textit{what} we are learning in this course and \textit{why} it is important.

I have been so focused on figuring out where to put the information in the lesson plan that I'd not considered the implications of this.  Each class should raise some considerations and concerns, and lead the students to discuss what they think or previously thought and the implications of this new knowledge.

\subsection{Particular Topics}

I had previously considered ethics such as Kaldor-Hicks and Pareto, with social choice being the override:  a group can decide to sacrifice some for the benefit of all, or to the point a person can decide to reduce their welfare so as to increase the welfare of tohse in greater need.  Kassner brought up two values when I mentioned this:  autonomy to choose to sacrifice for the general welfare, making the cost legitimate; and the utilitarian view, where the best outcome determines the correct choice.

He also raised the issue of persistent marginalized minorities, whose vote and opinions never really matter.  They're always on the losing end, unless a larger group decides to stand with them out of pity or duty or political utility, so is the democratic outcome truly legitimate?  I have suggested the closer to equal voting power a system becomes, the more legitimate it becomes \textit{specifically} because these groups put a lean on the outcome such that winning an election is \textit{always} contingent on minimizing the imbalance, or simply put if some small but significant group doesn't like you they can decide to elect the next-most-similar candidate who they would otherwise have liked less.

Kassner also raised that certain issues are not appropriate for the demos to decide.  From my perspective, this is with certain philosophical ideals, e.g. people are not property, voters cannot be disenfranchised, ideals cannot be verboten.  From his, the inappropriateness can stem from simply not being qualified to make a decision, in his example a school board composed of no scientists trying to decide that a thing considered pseudoscience is to be taught as science.  In that case, the court struck it down:  the jurisdiction had decided to teach intelligent design, a type of pseudoscience based in religion, as the scientific origin of the world, essentially State-established religion.

%Narrative arc - telling a story in the way you walk the class through the material
%  - McGrath?

% e.g. critical thinking:  not a walk through the content, but rather the development of arguments.

% Values - Autonomy, a person choosing to make a sacrifice for general welfare makes the cost to them legitimate; utilitarian, the best outcome is the right one.
% Persistent marginalized minorites - always on the losing end, so is the democratic outcome legitimate?  The marginalized are at the mercy of the powerful, rely on allies.

% Inappropriate for demos to decide certain issues; do they have the capacity to even understand the problem?

\subsection{High School Course}

I had originally intended to strip this course down and remove the deeper study into philosophical concepts, focusing on the mechanics of electoral systems, to integrate into a brief study during a high school civics course.  The basic consideration that the manner by which votes are accounted can, in a completely free and fair election where all voters are able to vote and all votes are counted, determine that a great many voters do \textit{not} in fact experience a free and fair election \textit{because their vote is rendered impotent} is key, and students should easily grasp the concept and its implications when given some of the mechanics.

Dr. Kassner suggested integrating the Federalist papers and some of Tocqueville, expanding to a full study of government, essentially creating a full civics course on a firm basis.  That is a project for the future, and one much more interesting than pitching that this concept must be taught in existing courses.

I may also wish to research the Federalist Papers for some of the college-level course; however, this is not a Foundations of American Democracy course or whatnot, and the broader historical focus should not hover around America alone.
