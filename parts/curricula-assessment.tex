\chapter{Assessments}

This class is meant to foster an interest in and understanding of the positive and normative justifications for liberal democracy, and of the mechanics of elections and government structures in their ability to uphold these principles of democracy.  Assessments should make students think, but not stress precise memorization of all associated facts.

In line with these goals, each lesson lists several associated multiple choice questions suitable for midterm and final examinations.  These avoid precise dates or other minor details, functioning more as reinforcement exercises.  They will be difficult to answer if not familiar with the material, and are aimed at being memorable and recognizable to students who have at least attended the class.

For midterms and final exams, short essay questions also provide a good exercise for students.  These are not rigorous, only intended to challenge the student to think and provide the instructor with evidence the student has understood the material.  Grading should be favorable to any reasonable argument demonstrating knowledge of the material, even if the student reaches different conclusions than what the instuctor might by applying (or misapplying) the principles learned here.

\section{Weekly Short Answer Questions}

If desired, the instructor can require one short answer question per week.  These should be of minimal complexity to reduce the grading effort required by the instructor, but of enough for the student to reflect on the material.

Short answer questions are listed as pertaining to the lesson given here, to be answered prior to the next lesson.

\subsection{Education, Knowledge, and History}


\subsection{Public Discourse and Good Collective Decisions}

\subsection{Collective Decisions and Consent}

\subsection{Majority Rule and Preferential Voting}

\subsection{Ethics and Welfare}

\subsection{Representative Democracy}

\subsection{Evaluating Decision Methods}

\subsection{Single-Winner Voting Rules and Elections}

\subsection{Multi-Winner Voting Rules and Elections}

\subsection{Government Structure}

\subsection{Electoral Manipulation}

\section{Essay Questions}

\subsection{Midterm}

\subsection{Final Exam}