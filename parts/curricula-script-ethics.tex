\refstepcounter{syllabuslesson}
\label{cur:ethics}
\section{Ethics and Welfare\hyperref[syllabus]{↑}}

This class discusses ethics in terms of economic welfare, and covers welfare-based voting systems.  Both straight Score systems and the implicit Borda rule are covered here.

Ethics needs to be placed next to justice, although Rawls's veil of ignorance works well enough.  For the purpose, Condorcet's reflections on negro slavery \autocite{Condorcet1781} work quite well, as he repeatedly points out the absurdity of the institution and, while making great concessions, that the slavers owe much more than is being demanded of them and that certain policies offered in response to all manufactured reasoning against emancipation are giving generous deference to the greed of the slaver.  Particularly, he indicates that all demanded is in fact owed to those emancipated, as the slavers have committed such a crime that they do not rightly own much of their property—long before Nozick's entitlement theory.

This becomes relevant when discussing Pareto, although Condorcet is attempting to exercise Scitovsky's Double.  He has a habit of using theories hundreds of years before they're initially proposed.

Reading Nozick goes too far from the topic.

This builds on weeks 2 and 3, as those two give a justification for the assertion that preferential systems—proper preferential systems that find the median voter—are high welfare.  Consider the absurdity of weeks 2 and 3 coupled with the assertion that any individual or small group of elites can be identified who are able to determine who should be elected for the greatest good of all, and it must be nonsense that the Condorcet winner is of high utility.

To really understand that, you have to understand public discourse as part of the democratic process; and to understand that, you have to understand the nature of knowledge and the difference between an individual's knowledge and the knowledge of the collectivity (i.e. Condorcet jury theorem and its implications).

In the development of policy, I often consider Pareto and derivatives, in that there must be compensation to ensure nobody is worse off after a policy change.  When this fails, I quickly fall back to Rawls's veil of ignorance.  I tend to avoid wanton taxation of the rich for this reason, unless I'm pressed; but due to the income effect I \textit{will} adjust the tax system to make it more-fair and more-progressive, again without arbitrarily using it as a punitive tool against people who have too much money, based on an assessment of both how much revenue is received in proportion when taxing those with less income and how much strain is on them by doing so, plus the veil-of-ignorance decision between reducing poverty and not punishing wealth.  It is an immense and complicated exercise in mental gymnastics, and this is the foundational knowledge I use when doing so.

\subsection{Assigned Reading}

\begin{boxcomment}
    Originally assigned \prettyref{apx:ethics}, covering Pareto, Kaldor-Hicks, Scitovsky's Double, and the income effect; then I found out we could get Tideman's book from interlibrary loan as an ebook.
\end{boxcomment}

Relevant reading assigned prior to this day:

\begin{itemize}
    \item Collective Decisions and Voting, Chapter 3 \autocite[23-32]{Tideman2006} covering Buchanan, Tullock, Rawls, etc. ethics theory from \hyperref[cur:collective-decisions-consent]{Collective Decisions and Consent}, along with the income effect.

    \item \prettyref{apx:ethics}, covering Pareto, Kaldor-Hicks, Scitovsky's Double, and the income effect

    \item \autocite{GreenArmytage2015} which finds the Condorcet winner is usually of high utility, close to the ideal welfare candidate

    \item A piece of bad research on score voting \autocite{Baujard2014}

    \item A better piece of research on score voting \autocite{Feddersen2009}
\end{itemize}

There is no reading from the Handbook of Social Choice.  It simply doesn't regard scoring rules for obvious reasons.

Baujard's paper favors score voting by reasoning that, using exit polls in which voters had no stake and finding little evidence of strategic voting, voters want to vote ethically but can't when using a runoff system, and would vote ethically if using score voting.  This is patently absurd:  it is the claim that every time someone says they would take an ethical action under some hypothetical condition, they definitely would take that ethical action under that condition were it to actually occur.  This is known to economists.

Federsen et al use an experiment in which voters receive a monetary benefit.  They find with higher pivot probability, voters vote in their self-interest; with lower pivot probability, voters perceive that their vote doesn't matter and won't change anything, and they vote for the good of everyone, even though this is directly portrayed as good for them, but not as good for them as the self-interested alternative.  This experiment uses real stakes with real consequences, unlike Baujard et al, and is much more methodologically sound besides.

\subsection{Overview}

\begin{todo}
    Consider the concept of Pareto and Scitovksy's Double in the context of slavery, women's rights, immigration, and other disenfranchisement.
\end{todo}

\begin{itemize}
    \item Introduction to and discussion of ethics including Pareto, Kaldor-Hicks, and Scitovsky's Double \autocite[24-30]{Tideman2006}

    \item Discusison of the income effect \autocite[30-31]{Tideman2006}, covers transitive and intransitive preferences more thoroughly

    \item An introduction to welfare-based voting systems
    \begin{itemize}
        \item Score voting

        \item Borda count
    \end{itemize}

    \item Discussion of welfare systems and ethical voting
    \begin{itemize}
        \item \autocite{Baujard2014} and \autocite{Feddersen2009}

        \item Excerpt \autocite{Reilly2002}, Borda count was immediately manipulated in Kiribati (not a discussion about how to manipulate, which is for Week \ref{cur:manipulation})

        \item Condorcet winner is often of high welfare; debate Condorcet vs. Welfare, in contrast with Feddersen and ethical voting behavior \autocite{Tideman2019}
    \end{itemize}
\end{itemize}

\begin{boxcomment}
    Pareto is mentioned when discussing unanimous consent via Rawls, Buchanan, and Tullock.  The normative concept of hypothetical unanimous consent requires this:  a pareto improvement would be good for all, and a Kaldor-Hicks improvement is better overall but may not be better for all.

    Scitovsky's Double allows decisions in which those made worse off may accept this so others can be made better off.  Under Rawls's veil of ignorance, a person would naturally consider the most ethical of trades here, because they may be the one much worse off.  This does not mean an equal wealth distribution:  the wealth—ultimately, income—of the rich, distributed among all, is small per each person, and so the economic hierarchy not only serves to maximize due to natural human tendencies but also provides little theoretical gain via flattening.

%    For example, with a minimum per-hour wage 60\% of the per-hour average wage, a worker may see an opportunity to climb from \$55k to \$92k; yet as well, the society may provide healthcare, unemployment insurance, public assistance for low-income households, and all manner of valuable things.  On the face, a 65\% difference is seen; but in total, it falls below 50\% considering marginal taxes, and further considering the relative value of the services provided by those marginal taxes.  This difference remains significant and sizable, but this exchange and the control of earnings inequality by binding the minimum wage to productivity (achieved by indexing to per-capita GDP) means the lowest-paid are paid well, share in the growth of the economy, have a strong social safety net, and have parts of their expenses paid for by society as a whole rather than by their individual and unequal wage.  Besides the economic efficiency, opportunity is created, with minimal losses for those receiving the least of it; thus the inequality is generally agreeable when the potential worst-case is generally acceptable.
\end{boxcomment}

\subsection{Content}

\begin{todo}
    Clean this up.
\end{todo}
Begin with a policy ethics discussion covering Pareto and Kaldor-Hicks; and then uses Scitovsky's Double and agreement on process as a callback to democratic legitimacy.  Kassner asserts democracy is not a process, but a set of principles; yet in a democracy, we agree on certain processes which must violate these principles to some degree.  Certainly we cannot vote away minority rights until we have a dictator; yet even in a Condorcet system, or any other system, some voters will dissent, or else unanimous consent is required and will never be achieved.  Key to this is that the voting process must maximize equality of voting power and may not strip voters of their power outright.

Having discussed ethics and welfare, we can discuss preferential versus welfare systems.  Preferential systems are already covered by instant runoff versus Condorcet systems, and these can be introduced in greater detail here.

This expands to welfare systems.  Score systems are easy to understand; they rely on voters voting honestly and accepting the outcome.  Higher scores also indicate higher preferences, so we can discover that the elected candidate was not supported by any majority.

Score voting goes directly to rational voter theory.  Voters decide whether to vote based on the expected utility to themselves \autocite[p54]{Heckelman2015}.  By extension, voters vote in their own interest.  It has been shown \autocite{Feddersen2009} that when voters expect a wide win margin, they vote honestly in score-based elections; but when voters expect an election to be close, they rate threatening opponents lower not because of their perceived utility, but because the whole point of score voting is that whatever candidate has the most utility is elected.  Rating a candidate as almost as good as your favorite can cause that candidate to be elected when your candidate gets a majority of votes, essentially on the principle that a minority of voters want their candidate more than the majority wants their own; if this is likely, then rating the candidate dishonestly low avoids the situation.

This opens the discussion of tactical voting, which is built on another day.

Scitovsky's Double bridges this gap:  in a ranked system, voters can rank a candidate above that which maximizes their own welfare.  This expresses a preference not to maximize their own welfare, but to maximize society's welfare.  In a score voting system, voters must rate honestly to discover what maximizes society's welfare; rating a candidate higher not for their own welfare, but someone else's, distorts these systems, which undermines the purpose of a score system.

Ranking a candidate as preferred even though they do not maximize the voter's own welfare is a voluntary agreement by the voter to bear the cost for someone else's welfare.  Such an agreement makes the transfer of welfare legitimate; but again, the body as individuals never wholly agree, and some are forced.
