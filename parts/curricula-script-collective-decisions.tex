\refstepcounter{syllabuslesson}
\label{cur:collective-decisions-consent}
\section{Collective Decisions and Consent\hyperref[syllabus]{↑}}

This session covers the taxonomy of collective decision functions, the complications therein, and the concerns of exactly how to start a democratic government.

This builds on knowledge and collective knowledge by discussing modes of collective decisions.  Tideman's taxonomy provides a technical component that happily breaks away from philosophy for a little while.

A problem arises in that agreement-on-outcome requires unanimous consent.  Democracies cannot form if we assume no power over anyone who does not consent to the agreement-on-process mode.  There is also the question of holding them and all who follow them to their consent, which cannot be withdrawn without the agreed-upon process allowing them to do so (this would expand to the Civil War and the authority of the Union to prohibit States to exit).  While immigrants will have automatically consented by entering the nation and becoming subject to its laws, children are only born into someone else's agreement made without their knowledge or consent.

Buchanan, Tullock, and Rawls provide the framework of Constitutional consent.  A rule which would be agreed upon under certain rational circumstances is said to be agreed upon unanimously.  This leaves questions easily dismissed:  many alternative rules would be acceptable in this framework, so a particular rule is not unanimously accepted; however, fundamental ideals of equity are unanimous in the framework, and so we can say there is unanimous consent to the rule of selecting \textit{some} satisfactory rule which we can unanimously \textit{accept}.

%, and how we justify democracy and voting if we need unanimous consent.  There may be a lot of filler, but it's a good topic to fill…perhaps not that good, though, as this is a lot of space to fill.

\begin{todo}
    Incorporate Rawls's concept of a democracy as a joint venture?
\end{todo}

\subsection{Assigned Reading}

Relevant reading assigned prior to this day:

\begin{itemize}
    \item Chapter 3 of the Handbook of Social Choice and Voting \autocite[35-51]{Heckelman2015}

    \item Is Everything Really Up for Grabs? \autocite{Kassner2006}

    \item \prettyref{apx:taxonomy-of-collective-decision-procedures}, covering the modes of collective decision making \autocite{Tideman2006}

    \item Condorcet's reflections on the enslavement of negroes \autocite{Condorcet1781}

    \item ``On the Admission of Women to the Rights of Citizenship'' \autocite{Condorcet1789}
\end{itemize}

\begin{boxcomment}
    I considered leaving Condorcet's writings on the rights of women for the next class to facilitate debate in this session, as the arguments can be used to challenge students for a response; however, the arguments are of a nature as to be impossible to naturally reason out without the devotion of time, study, and deep thought on the matter.
\end{boxcomment}
\begin{boxcomment}
    This is a good place to breach the normative question of whether democracy is a process or a set of principles.  This already started with Week 1, discussing the three possible winners based on the voting process, but I can't assign reading prior to the first day of class.  Consent implies principles, not process.
\end{boxcomment}
\subsection{Overview}

\begin{todo}
    Again, slavery and minority voting rights, women's voting rights, and felony disenfranchisement.  Citizenship is interesting:  when immigrants are here and become residents, is it legitimate to not automatically confer citizenship after a short period of residence?  Consider State citizenship, which is immediate.  Few people can immediately obtain residence and then leave, even under open borders policies.
\end{todo}
\begin{itemize}
    \item Begin with a short discussion about the taxonomy of collective decision functions, and how democratic society uses authority and voting

    \item This is a good place to breach whether democracy is a process or a set of principles

    \item Once the discussion is seeded, lead into consent to Constitutional rules, centering on Rawls's veil of ignorance and Buchanan's model by which the rules are better

    \item Students should naturally recognize that these models presume consent by a person who \textit{would} consent if their situation were different; else bring this up in the discussion

%    \item Lead the discussion to tie together public discourse, Condorcet jury theorem, and consent:  a democracy essentially attempts to bring together all perspectives to discover this veil-of-ignorance perspective

    \item Devote at least half the class time to discussion and debate over the writings of Condorcet and the legitimacy of government when women, blacks, and felons are disallowed their political rights; and in general about the U.S. Constitution and its original form, versus Rawls's veil of ignorance
\end{itemize}

\begin{boxcomment}
    This ties together public discourse, Condorcet jury theorem, and consent:  a democracy essentially attempts to bring together all perspectives to discover this veil-of-ignorance perspective.
\end{boxcomment}
\subsection{Assessment}

\subsubsection{Multiple Choice}

\begin{itemize}
    \item \multiplechoice{Who proposed the theory of unanimous conent based on a ``veil of ignorance'' during the making of constitutional rules?}{John Rawls}{James Buchanan}{Jeremy Waldron}{Condorcet}


\end{itemize}