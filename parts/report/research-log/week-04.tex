% vim: sw=4 ts=4 et

\section{Week 4}

Dealt with backlog of maintaining this research log since Week 2.  This is a problem; best to keep on top of it going forward.

\subsection{Literature Review}

\subsubsection{Philosophy of Legitimacy}

Re-read Estlund \autocite{Estlund2008}, Rawls \autocite{Rawls1997}.  Also drew in some research related to score voting and ethical behavior \autocite{Baujard2014,Feddersen2009}.  It occurs to me some of the writings of Marie Jean Antoine Nicolas de Caritat, Marquis de Condorcet, are relevant, particularly ``De l'admission des femmes au droit de cité'' \autocite{Condorcet1789}:

\begin{quote}
    But the rights of men result simply from the fact that they are rational, sentient beings, susceptible of acquiring ideas of morality, and of reasoning concerning those ideas. Women having, then, the same qualities, have necessarily the same rights. Either no individual of the human species has any true rights, or all have the same; and he or she who votes against the rights of another, whatever may be his or her religion, colour, or sex, has by that fact abjured his own.
\end{quote}

Condorcet shows that any who claim one person does not have certain rights must necessarily claim that no person, including themselves, has those rights.  He extends this from women to religion, race, and any other qualify by which we may differentiate members of the human species.

The form of his argument goes so far as to include any sentient, intelligent being, although it's doubtful he entertained the existence of such people.  Nevrtheless, in 1873, a full 84 years later, Francis Galton misused Darwin's theories of evolution to argue for the elimination of ``the inferior Negro race,'' while others acting under the guise of ``science'' argued that blacks were a separate species\footnote{What is this even? \url{https://www.ferris.edu/HTMLS/news/jimcrow/letters/2012/apes.htm}}.  Even if any of this were true, being in the possession of such characteristics as convey all the rights of the human species, such rights would equally apply to them as well—a powerful rebuttal in that disproving the presented assertions is not a prerequisite to proving the possession of human rights.

\subsubsection{Score Voting}

Read two papers on rated voting systems.  One examined range voting in the French 2012 election \autocite{Baujard2014}, and the other used an experiment to identify ethical behavior \autocite{Feddersen2009}.

Baujard et al was striking.  The authors conducted exit polls, asking voters to fill out score voting ballots, and looked for tactical voting patterns.  They found very little, and so concluded that voters \textit{want} to vote ethically, but that the plurality runoff system prohibits this.  There is a huge error here:  there was no stake in voting unethically.  All economists understand that when given a decision, a person will act based on utility:  if the marginal benefit is zero, then it's not worth doing.  There's no benefit to tactical voting on a meaningless ballot.  The conclusions were not just poorly supported, but intellectually dishonest.

Feddersen et al used an experiment with a monetary payoff.  Participants were arranged into two groups.  None were told who else was in their group, but they were told which group they were in.  Group B was allowed to vote; one ballot was randomly chosen to be the dictator, controlling the outcome.  In this way, the researchers reported to the voters when their individual vote was pivotal.  When the chance of a pivot was lower, voters voted ethically; but when they believed their vote had a high likelihood of actually changing the outcome, they downrated the alternative that served their self-interest less, choosing to maximize their gains rather than maximizing the total gains of the whole group at the expense of having slightly less themselves.

Feddersen's conclusions can be read in one of two ways.  One may suggest that in a large election, a person's vote is not seen as pivotal, and so voters will never vote strategically; this has been proven untrue elsewhere.  By reason, however, we know that voters \textit{shouldn't} vote as often as they do in terms of our expectations given their marginal utility:  their vote isn't likely to be pivotal, so they should not expect their vote to matter, \textit{yet they do}.

In a close election, voters perceive the group:  they talk about how they need to get out to vote so they can win.  They expect others to do the same.  They see their own vote less, and instead become part of an expected mass of votes cast with theirs.  A close election thus appears pivotal to voters, and voters would expect others to vote the same as they do; the consequences of voting are amplified, as are the gains from voting tactically.  Contentious elections would shift closer to plurality, where ballots give one candidate the maximum score and other candidates close to no score.

The two studies are striking and show why score voting isn't taken seriously in academia.
