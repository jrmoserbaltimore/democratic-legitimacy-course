% vim: sw=4 ts=4 et

The University of Baltimore has approved construction of a course covering social choice, electoral methods, and democratic legitimacy as an independent research project for Spring, 2021.  My understanding of elections and voting rules is insufficient to produce course curricula; and I did not begin with a background in philosophy and democratic legitimacy.  Fortunately, Dr. Kassner has chosen to act as my mentor in this undertaking.

This report details the development of the curriculum, the research done, and the lessons learned along the way.  I intend that readers already understand the subject matter, possibly by first completing the course produced in this effort.

One of the biggest lessons learned here is this is truly a research project, as is producing a textbook, and requires literature reviews beyond simply finding material covering the topic.  Were it so simple, a quick sketch of a course outline, some time double-checking facts I already know, and simply writing out the entire thing would suffice, complete in a matter of days.  Here instead I have learned things I did not know about things I already knew quite well.  One cannot simply write down what they know and call it a curriculum, as it will invariably be incomplete if not inaccurate.

The first part explains the conception of the project, beginning with the personal experiences which convinced me such a course was necessary; following with the proposal; and concluding with a short overview of what I produced as a result.  The second part contains the curricula, describing the contents of each weekly lesson and the supporting materials used; the curricula is not a tool for self-teaching the subject, in which instructors must already be well-versed.

The appendix includes a meeting log and a research log, containing simple notes.  It also contains a list of select voting rules and reading material assigned throughout the course.
