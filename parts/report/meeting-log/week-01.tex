% vim: sw=4 ts=4 et

\section{Week 1}

First meeting with Dr. Kassner to discuss the project's goals.

We decided a report on the process would be appropriate, as that's really the part I'm learning—most of it anyway.  Learning more than I already know about social choice is unavoidable.

Dr. Kassner raised the importance of normative considerations.  Does democracy even matter?  What do we believe is the goal?  Do we want to elect trustees who make important decisions from their experience, or delegates who do what we demand?  Is direct democracy better, or does it create injustice by the shifting will of majorities and their power over minorities?

I insist democracy is a set of principles, not a process; this was clarified by Dr. Kassner's own arguments, and the normative questions are for the students and generally bring me no concern.  Still, just as others seem confused by questions like ``when is voting democratic,'' I am unfamiliar with some of thoe he asked.  My observation has been a brief discussion of the mechanics of voting systems quickly convinces people that an electoral process can suppress the voters and thus be anti-democratic; thus far, these new questions are challenging, but only insomuch as I have to answer them:  they have only reinforced my understanding.

The important one was the structure of government.  I have proposed adjustments to how we view and operate bicameral legislatures as a means to maximize both representation and diversity of ideals.  Dr. Kassner raised questions about whether a government should be representative at all, and by trustees, or if it should be a more direct democracy.  I recognize the problems in direct democracy and subscribe to elections which are free, frequent, and fair, maximizing voter control over who is elected so that unsatisfactory trustees can be quickly replaced, but still act as trustees with their greater suitability.  The trustees are servants aiming to solve our problems in a satisfactory way, and our abilities as voters are limited to judging whether these servants are working effectively in our interests.

That is a utilitarian judgment:  the majority one week may decide all immigrants must be deported, and then a small number of voters change their mind and decide we should give them all citizenship.  This is unstable and dangerous, and while one side might actually be right, both sides are poorly informed and incapable of knowing they're right.

\subsection{Knowledge}

What is knowledge?  How do you know something?

Three criterion:

\begin{enumerate}
    \item It must be true;
    \item You must believe it is true;
    \item You must have a valid reason for believing it is true.
\end{enumerate}

The first can never be measured:  empiricism relies on measurements, observations, and logical reasoning to be true.  You might just be insane, or literally everyone could be wrong about something.

The second requires belief.  If you lie about something and claim it is true, you don't know it is true.  If it turns out to be true, you had believed it false and only pretended it was true.

The third is subtle.  Believing the earth is round because space aliens abducted you and flew around the planet several times doesn't count:  even though you're right, there are no space aliens, and you have no other reason to believe the earth is not flat.

That the majority are actually correct about something sometimes does not mean they have any idea what they're doing.  When majority coalitions shift back and forth, creating an unstable government, the side that's actually right is not necessarily any better than the side that's wrong, in terms of knowledge.  Believing minimum wage is just and fair is not a reason to believe minimum wage does not reduce employment; there is both a great deal of empirical evidence and non-published theory\footnote{I'm working on a paper on my theoretical research into minimum wage, and the laws of both mathematics and physics would need to be invalidated to prove the theory wrong.} showing minimum wage does not reduce unemployment.
