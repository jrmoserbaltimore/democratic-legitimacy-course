
\section{Week 8}

Conversation about Buchanan-Tullock and Rawls writings on Constitutional Consent.  Kassner pointed out Rawls also describes democracy as a mutually beneficial cooperative venture.

I brought up that a lot of the readings I'd want to assign \textit{after} the class, which is strange, but in the philosophical end I want to draw debate out of people instead of making them repeat what they've been told.  He pointed out that this can get out of hand and it can be hard to steer the conversation, although I'm relatively confident I'd be able to do that.  In debates I tend to steer people where I want them to go, essentially controlling both sides of the conversation, once I have a feel for their arguments and how they respond to prompting.

I'll keep a few things in reserve, but there are plenty of things I can't draw out organically.  I'll think of something.

Interestingly, we discussed the approach and tone of the course.  I want to use a method of hostile debate:  I tend to understand my opponent's likely arguments and lead them to commit to unsupportable claims, particularly claims I can thoroughly destroy in few words, or in a manner that breaks down into small and easily-followed pieces strung together.  The plan has always been to open with questions about elections and who should be elected from examples, rapidly undermining every bit of the student's beliefs one after another.  We discussed an enhancement on this.

The first class is as aggressive an opener.  Rather than telling students what to think and having them nod along and absorb, I draw out of students a commitment to a belief upon which they've founded their sense of fair and just society.  I make them think about it, acknowledge it, bring its shape into focus in their mind, and then provide them new information that exposes all the cracks until it collapses in a heap of rubble.  This happens at a time when the student has reached for that belief, has affirmed to themselves its importance.  At this point, what has been seen cannot be un-seen.

The course gives students the tools they need to understand how to protect their ideals, how to recognize the dangers to democracy.  The goal is not to terrorize the students, but to create mistrust and suspicion of anything purporting to be democratic, and give them the tools they need to examine it.  Questions should draw hesitation from students who think they have answers as they mentally search for reasons those answers might be faulty.

As an example, we discussed the current push to eliminate the filibuster in the Senate.  This is popular, and purported to be ``democratic'' because of ``majority rule'' and the perception of a small minority vetoing the majority.  Students today would likely enthusiastically support its elimination; at a certain point in this course, students prompted with the question will consider the reasons \textit{why} they support its elimination, and doubt their own position, thus hesitating to answer.   Before fully exploring the two-tier problem, the students will not have an answer at hand, but may have some suspicions due to knowledge of Condorcet systems versus instant runoff voting; the two-tier problem then shows that a minority of voters are represented by a majority of Senators, and the discussion of k-majority rule sits on firm context and prompts a great deal of consideration about the filibuster.
