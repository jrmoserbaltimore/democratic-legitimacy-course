\chapter{Lessons Learned}

I hold a CAPM; how I didn't think to write a lessons-learned document before sending this in is beyond me.

Much of this is about how I did research.  Completing this is going to be a several step process, based on these lessons:  first, the general proofread to get everything into some workable half-completed shape; second, an actual, thorough research project, using this as a starting point and producing a plan to modify and improve on it; third, reworking each lesson based on this project, focusing on one at a time; fourth, another proofread.  Repeat as necessary.

\section{What Worked}

\begin{itemize}
    \item \textbf{Organize the narrative at a high level}.  Dr. Kassner suggested I start with a course narrative.  I organized a narrative of flow through \textit{concepts}.  That was what broke the main blockage preventing me from getting this in shape at all.

    \item \textbf{Track readings and concepts}.  The narrative became easier to put together and the project as a whole started to run much more smoothly when I started tracking what reading went with what lesson, why it was introduced there, and in what future lessons the reading and concepts within were referenced.  This also affected the content substantially.

    \item \textbf{Read and reread}.  I reread some of the material, notably the Handbook of Social Choice and Voting \autocite{Heckelman2015}, several times.  Much of the material was familiar, especially after the first reading.  Even so, covering the chapter I'm working on helps.  Eventually I started recognizing theorems like Plott's Majority Equilibrium Theorem, which I didn't initially put into context.
\end{itemize}

\section{What Didn't Work}

\begin{itemize}
    \item \textbf{Starting with a complete narrative}.  I tried.  It didn't work.  Describing every individual lesson in detail and how I want the course to flow is not different from writing the course.  Organizing at a higher level, as described earlier, made the narrative work out much better.

    \item \textbf{Starting by doing}.  I sat down and started just writing, imagining how I'd teach different lessons.  This doesn't work.  Without an overall plan, the lessons are incoherent.

    \item \textbf{Using material as references}.  I read the philosophy papers; I skimmed the parts of the Handbook that were familiar.  I pulled out the material to find things I wanted to reference and to pick out individual concepts and get a brief refresher or expansion of a concept along the way.  This worked for the papers I read, and didn't work for the books I was just referencing.  It got better when I read the material in full instead of picking at the pieces I wanted.

    \item \textbf{Assignments}.  Don't write up tests and quizzes and essay questions until the damned thing's finished.  They're the least important thing:  the material can always be turned into assessments.
\end{itemize}

\section{What I'd Do Differently}

\begin{itemize}
    \item \textbf{Actually read the material}.  Early on, I started laying out the lessons, searching through the chosen textbook for the concepts I wanted to teach and for unfamiliar but related ones, and piecing the course together.  \textit{Then} I actually read a whole chapter.  It was fascinating; also it conveyed a \textit{lot} more than the concepts I had pulled out of it.  The text surrounding these concepts, which I thought I understood already, put them into a bigger, more complete context.  Supermajority is a good one:  I understood the danger of slim majority votes in the U.S. Senate, and had continuously argued that the 60-vote cloture rule was of critical importance to avoid the tyranny of the majority; a thorough understanding of Buchanan-Tullock reinforced this, but it was far from a simple rehash of what I already knew and substantially altered some parts of this course.

    \item \textbf{Take notes on what I read}.  After the initial annotated bibliography, I started notating papers that were of interest, but not taking too much down about what I read.  Relying on the paper to be its own later explanation proved a mistake.

    \item \textbf{Track what the readings are about}.  Not just notes about what I read, but very short notes about the useful points from a chapter or paper.  Theorems, concepts, names, and how they tie to different lessons or the conceptual idea behind what might become a lesson, separate from notes about what any of those things mean.

    \item \textbf{Start with an actual research project}.  I researched while writing this.  Shifting around the lessons, notating the associated readings and the purpose attached, and taking down simpler notes about what I'd read and the basic intent of each lesson would have been much more efficient.  I wouldn't have reordered and rewritten so many lessons, or needed to proofread a few times during the project to make sure lessons which had been moved earlier in sequence didn't still reference later concepts.

    \item \textbf{Set aside concepts}.  When adjusting a lesson, I would think of some new thing and stuff it into another lesson, or remove something from the lesson and immediately transfer it into another lesson.  I should keep these as separate notes to be reviewed and executed along the way.
\end{itemize}