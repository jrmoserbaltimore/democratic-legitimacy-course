
\chapter{The Result}

I began this independent project with a simple goal:  create a course to teach about voting theory, elections, and democracy in the technical context.  Dr. Kassner suggested integrating democratic legitimacy, and provided a few pieces of research on that topic.  Even then, this project went in directions I had never considered.

My first attempt had focused on avoiding too much philosophy up front, aiming for a mix of technical and philosophical study to keep non-philosophy students from becoming disinterested.  That created all kinds of questions and difficulties; it didn't quite work out.

As I researched and planned, I began sequencing lessons, with a large amount of difficulty deciding what to teach, how, and in what order.  Dr. Kassner suggested I create a narrative and use that to guide the course; this did not work.  Nevertheless, as I went back to sequencing the lessons, I started seeing a narrative \textit{in} the lessons, building each piece on prior knowledge, setting up debates by the sequence of concepts covered, and weaving threads into the narrative such as a focus on women's rights, slavery, and felony disenfranchisement.  Once I started looking at the subject matter as its own narrative, rather than looking at the two as separate things, it became easy.

This shaped the course heavily around philosophical debates early on, filled with new knowledge I'd acquired during the early parts of this project.  The simple concepts of Rawls's public discourse, Estlund's epistemic democracy, and Constitutional consent theories by Buchanan, Tullock, and Rawls created a foundation supporting the principles of democracy and provided tools to evaluate policy decisions in a democratic society.  Whereas I had first considered legitimacy from the perspective of whether every vote was equal, I now had Rawls's veil of ignorance to go along with Kaldor-Hicks and Pareto theories.

Further research unearthed ethical behaviors in studies about score voting, and new voting rules such as STV-B (Geller-STV).  I am uncertain how resistant Geller-STV is to manipulation; there are important logical propositions suggesting manipulating the Borda scores comes at the expense of guaranteeing a worse outcome based on election by vote count.  This is interesting because STV, reduced to one winner with a quota of 50\%+1, becomes IRV, which is highly manipulable; Borda is \textit{ridiculously} easy to manipulate; but IRV-B seems to be highly-resistant.  With three candidates, IRV-B elects the Condorcet winner; this does not extend further, but representativeness is higher and may translate to Geller-STV.

Other considerations arose as I worked on this, including the merits of taking an entire lesson to examine propaganda from organizations advocating various electoral methods, and on the manipulation of a referendum by controlling which options are presented.  I started explicitly identifying traps to work in for debate, leading students to incorrect conclusions and then making these obviously incorrect by new information provided in a future lesson.  This project is, at its heart, a course on identifying propaganda in a particular topic, specifically claims that a current or proposed democratic process is actually democratic.  Emphasizing this by exposing students to the consequences of falling just a little short of having all the information seems valuable.

These will require further effort, as although it's competently framed out, I have not completed the course curricula even by the end of this semester.  Even as-is, there are notes everywhere, hastily-written sections, and significant exmaples of bad grammar; it needs a major proofread, which is no small task.  I will send further updates to the school as I work on improving the curricula; and I have decided to expand this to a high school curricula for American government and civics education.  This was a much larger project than I had anticipated, and a much more valuable one:  I had intended to provide an important tool to give students a new way of thinking about democracy, and I came out with my own new ways of thinking about democracy.
