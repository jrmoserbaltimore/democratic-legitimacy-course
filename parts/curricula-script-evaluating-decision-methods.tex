\refstepcounter{syllabuslesson}
\label{cur:evaluating-decision-methods}
\section{Evaluating Decision Methods\hyperref[syllabus]{↑}}

Discusses criteria for evaluating collective decision procedures.  Are they efficient?  Equitable?  Stable?  This covers outcome efficiency, which determines if there is a consensus or if the system doesn't reflect the voters.

This builds directly on Tideman's taxonomy of decision making methods.  There is a small bit of conceptual consideration which raises enormous, unending questions and the entire class will be high-quality debate.  That voting can cause inequality and unjust results due to the tyranny of the majority runs up against the Condorcet winner being selected by all majorities, and the median voter.

I still want to examine this for more content and really need to dig into this week's lesson.

\subsection{Assigned Reading}

\begin{todo}
    Put together a complete reading summarizing Tideman's explanations.
\end{todo}

\begin{boxcomment}
    This stuff is based on chapters 4-5 of Collective Decisions and Voting \autocite[35-56]{Tideman2006}.  I was going to summarize in \prettyref{apx:evaluating-dm}, but the ebook is available via interlibrary loan for free.
\end{boxcomment}

\begin{itemize}
    \item Collective Decisions and Voting, chapters 4 and 5 \autocite[34-56]{Tideman2006} or a substitute reading
\end{itemize}

\subsection{Overview}

This is seriously just a three-hour philosophical discussion.  There's nothing to teach here; the concepts presented by Tideman tie into everything already taught, and we can debate the various forms of efficiency, equity, and stability against concepts of ethics and constitutional consent for literally centuries.

Occasionally, someone injects into a conversation that "more democracy is always better," which may become relevant here.