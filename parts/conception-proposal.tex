% vim: sw=4 ts=4 et

\chapter{The Proposal}

The obvious solution is to broaden the teaching of social choice theory, or the important basis.  What exactly that means iwas not initially clear; the final form came into focus as I developed the course.

Eventually I considered proposing an independent study to produce the course materials.  I had previously done work at CCBC for a professor developing some materials—but not an entire curricula—as a substitute for a core course I clearly didn't need, so the idea has some precedence.

The course was to have no prerequisites.  The material is so easily approachable that it could be folded into a ninth grade civics course—the American Government course taught at my high school was so thin it could be taught in three months, and a college course worth three credits \textit{is} taught in three months.  Even at a high school course, with 45 minutes per day five days per week, can cover basic voting theory in scant weeks.

This effort requires a large amount of research.  There are other courses out there, several books, and many considerations about exactly what to teach.  ``Social Choice Theory'' sounded good when I started, but Dr. Kassner had suggested democratic legitimacy as a topic, and provided materials I found relevant and unexpectedly interesting, going so far as to clarify my underlying motivation:  democracy is not a process, but a set of prinicples to which process must conform as far as is achievable.  This course will inform students of \textit{that}, and explain through social choice theory what it means.

At a college level, with three credit-hours in the course, open discussions with the class involving the philosophy of democracy, government structures, and history are possible.  A high school curricula may simply introduce the questions ``what is a majority?'' and ``when is voting democratic?'' and then discuss technical concepts such as two-tier representation, Condorcet criterion, and a few voting systems, enhancing the general discussion of American government.  A college curricula is necessarily more dense, and can cover discussions of collective choice, public choice, a thorough review of voting system criterion, published literature, Condorcet's Jury Theorem and its implications more generally in democracy, and so forth.  All of this is teachable at the high school level, but is not necessarily as valuable as a different focus.

That became the proposal:  some vague idea of teaching students about the underlying foundations of democracy itself, put into practice, in a way which only requires the student to be vaguely familiar with the concept of voting, or to understand the ten-second explanation.
