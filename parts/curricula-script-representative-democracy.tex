\refstepcounter{syllabuslesson}
\label{cur:representative-democracy}
\section{Representative Democracy\hyperref[syllabus]{↑}}

This covers representative democracy and justifies why we have it.

This is a good place to cite Jeremy Waldron about judicial review and ask questions about the authority of judges and executives, and even the authority of legislators—things automatically accepted in Representative Democracy.

\begin{boxcomment}
    Brings together representative democracy, which touches on everything taught in this class so far.  I was sure Rawls had a theory of justice mentioning the equal right to hold office, but I'm less certain about anything that specifically claims representative democracy is good.  Touch on the two-tier problem here to keep open the question about representative democracy, what constitutes a majority, and when voting is democratic.
\end{boxcomment}

Representative democracy is legitimized by the democratic choice to elect officials via an agreement on process, and the economic consideration that people cannot devote their time to understanding all topics.  Without immense specialization, individuals rely on the public discourse to understand the problems in society and the merits of various approaches to address them.  This at best helps identify the best individuals specializing in solving public problems, rather than a specific solution in detail.

These considerations are sensible, and fit both Buchanan-Tullock and Rawls's veil of ignorance.  Rawls asserts that differences in a person's station in society by way of elite public office is just so long as all persons have equal opportunity to those positions\footnote{Which paper was this?}, which is consistent with the veil of ignorance:  the advantages of representative democracy are unacceptable unless all have equal opportunity, by the will of the whole of society in whatever manner, to be elevated to office.  Likewise, the ability to remove bad representatives would suggest frequent, free, and fair elections.

\subsection{Assigned Reading}

Relevant reading assigned prior to this day:

\begin{itemize}
    \item Chapter 7 of the Handbook of Social Choice and Voting \autocite[102-114]{Heckelman2015}

    \item Chapter 8 of the Handbook of Social Choice and Voting \autocite[117-136]{Heckelman2015}\footnote{A priori voting power}

    \item Jeremy Waldron on judicial review \autocite{Waldron1998}

\end{itemize}

\subsection{Overview}

\begin{todo}
    Should visit a priori voting power here, revisit the two-tier problem, and examine Greenberg's core existence theorem \autocite[179-180]{Heckelman2015} in terms of K-majority in the U.S. Senate and its relationship to bicameral legislatures in restricting the core to a narrower unbeaten set.
\end{todo}

\begin{todo}
    This isn't filled in and doesn't cover the topic yet.
\end{todo}

\begin{itemize}
    \item Direct democracy has huge economic costs, due to voters not being experts.
    \begin{itemize}
        \item Knowledge \autocite[p.57]{Tideman2006}
    \end{itemize}

    \item In government theories, representatives serve several purposes
    \begin{itemize}
        \item Moderate rapidly-changing voter sentiments (\emph{DOMA, Maryland's anti-insurrection bill})

        \item Elite experts to traslate voter need into policy action

        \item Lead voters by the ideals they believe will most benefit them
    \end{itemize}

    \item Representative democracy involves complex application of Tideman's taxonomy of collective decision procedures
    \begin{itemize}
        \item Society selects legislators by agreement on procedure, specifically voting; and these make legislative decisions by an agreement on procedure, which includes voting.

        \item Society selects executives by voting, and executive decisions are by agreement on procedure—specifically, authority.

        \item Executive appointments are by authority, but society frequently agrees to a procedure whereby an elected legislative body must vote to accept the appointment.

        \item Judges are frequently appointed, and exercise authority over sentencing; trials are by an agreed procedure, which includes appeals to the authority of other judges, and the right of the defendant to instead appeal to a vote or unanimous consent of a jury of peers.  In many States, judges must achieve reelection after several years of service, appointed by executives based on extremely specialized ability to identify good judges (by their own judgment) and accepted by voters after observing the judge's performance—or rejected for egregious misalignment with society's values.

        \item Frequent, free, and fair elections allow voters to replace representatives as necessary.
    \end{itemize}

    \item Representative democracy has epistemic utility and Constitutional consent
    \begin{itemize}
        \item Representative democracy enhances the public discourse by focusing on general problems and solutions offered by dedicated public officials elected or appointed to bring together available information from specialists and the public discourse.

        \item Constitutional consent such as by Buchanan-Tullock or Rawls's veil of ignorance would support representative democracy.

        \item A procedure exists to change these basic rules via amendment of Constitutions, including City charters.
    \end{itemize}

    \item Representative democracy affects a priori voting power
    \begin{itemize}
        \item Two-tier democracy can severely diminish voting power of voters translated through legislators
    \end{itemize}

\end{itemize}
