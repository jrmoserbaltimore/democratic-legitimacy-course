\refstepcounter{syllabuslesson}
\label{cur:government-structure}
\section{Government Structure\hyperref[syllabus]{↑}}

Brings this all together with government structures.  This lesson addresses topics like k-majorities and the US Senate filibuster, single- versus multi-member districts, and City governments.

The case study used for Maryland State government retains the structure of government but alters the voting rules to engineer something completely different in its basic design.  This demonstrates a diversity of voices from each district in the House of Delegates, and a consensus of each district in the Senate, and in the Executive.

The case study for Baltimore City \textit{completely} re-engineers the government, and the merits of elected versus appointed executives should be compared more carefully to not evagicalize a proposed government structure.  That structure is one I actually proposed in a modification of the NCL's Model City Charter.  There are interesting interactions and implications for the Council body, and I tailored the specific details to distribute the ``power of corruption'' to increase the amount of collusion required to pull off what an elected Mayor can do in backroom deals, making such collusion dangerous, hard to keep secret, and unstable across elections.  It's not just ``we need our votes to count,'' but a whole examination of what the government structure accomplishes and how the method of voting and representation affects that.

\begin{boxcomment}
    Supermajority rule and bicameral legislatures have a similar effect to k-majority rule in making the vote more demanding and reducing the number of possible values to fit greater consensus \autocite[180]{Heckelman2015}.  The U.S. Senate has unique powers in approving treaties and executive appointmetns, notably including Supreme Court Justices; becaus this circumvents the decision-making complexity of a bicameral legislature, k-majortiy has large democratic merits.  This is hotly debated today; if you want to convince the students of this, you'll have to give them the tools to reach the same conclusion—pry at them to make sure they're not simply letting you dictate what they should think.
\end{boxcomment}

\begin{boxcomment}
    See ``The Misguided Renaissance of Social Choice'' by Maxwell L. Stearns, arguing that Madison and Jefferson never read anything by Condorcet about the Condorcet paradox or unicameral legislatures.  Interesting, but not addressed.

    This piece has an interesting examination of normative proposals; two quotes stand out:

    ``Scholars have used Arrow's Theorem to argue that legislative bodies are fundamentally incompetent, justifying significantly expanded judicial review.''

    Jeremy Waldron would have an aneurysm.

    ``Frank Easterbrook and others, for example, have observed that appellate courts, including the Supreme Court, are collective decisionmakers subject to the tenets of social choice.  This insight undermines arguments for expanded judicial review that are premised on the ground that legislatures are uniquely susceptible of cycling.''

    This is not necessarily true.  The Justices, as with the legislators, are a voting collectivity.  Legislators are elected in a process exposed to cycling; they are also prone to cycling, but usually get binary choices.  Recognizing considerations of social choice is not inconsistent particularly because of k-majority impacts on decision versus outcome costs, and there is a particularly strong argument for requiring a supermajority of Justices to make a Supreme Court decision.
\end{boxcomment}

\begin{boxcomment}
    Fun fact:  Nevada has a unicameral legislature.  Studying its performance is an extremely complicated matter requiring heavy statistical analysis and all kinds of skills you'd probably only learn in graduate school.  This course would prepare students to understand the resulting research, but not to carry it out themselves.
\end{boxcomment}

\subsection{Overview}

\begin{itemize}
    \item Bicameral legislature
    \begin{itemize}
        \item Condorcet argued that unicameral legislature was more efficient than bicameral and that bicameral gained nothing (Possibly move to the history lesson?)

        \item Bicameral legislature increase the complexity of decision making, similar to k-majority rules with $k>50$.

        \item U.S. Senate has unicameral confirmation of treaties and Executive appointments.
        \begin{itemize}
            \item Consider the historical margins for confirming Supreme Court Justices versus the recent 50+1 rules

            \item This is a good time to revisit the two-tier issue, considering the Senate's dissimilar-sized districts (States) with equal numbers of Senators
        \end{itemize}
    \end{itemize}

    \item Alternative construction:  Maryland State Government
    \begin{itemize}
        \item House and Senate are currently single-district, one Senator, three Delegates by MNTV, all 4-year non-staggered terms

        \item One proposal has been dividing Senate districts into three Delegate districts, rather than one multi-member Delegate district
        \begin{itemize}
            \item Arguments against multi-member districts have focused on racist motives to minimize minority votes

            \item Impact of various electoral methods, e.g. plurality, Final Five, STV nominating primary with Condorcet election

            \item Maximum representation means each Delegate is a geographical median voter, each Senator is the geographical median voter from three Delegate districts
        \end{itemize}

        \item Another proposal has been STV for Delegates\footnote{Eric Luedtke once brought this up in a private conversation.}
        \begin{itemize}
            \item Discuss prior argument about multi-member districts

            \item Compare MNTV with STV in multi-member districts

            \item Students should discuss the merits of Condorcet single-member Delegate districts versus STV multi-member Delegate districts

            \item The effects of gerrymandering are significantly different between these two
        \end{itemize}

        \item Bringing it all together:  engineering a government with heavy consideration of the voting rules.  Cajole students into exploring how voting rules can be leveraged as such.
        \begin{itemize}
            \item Condorcet Governor and Senators, STV multi-member Delegates

            \item The bicameral legislature creates complexity to improve the quality of legislation

            \item The Senate is a moderating voice, representing the diversity of views of Marylanders geographically, but within those geographic delineation represent consensus

            \item The House represents the greater diversity of views of Marylanders, not ony geographically but within geographies, creating contention and leading thought by elevating minority voices further than a Condorcet system can

            \item Compare 4-year House/Senate terms with 2-year Delegate, 4-year Senator terms as in most States; staggared Senate elections; and 4-4-2 staggering as in states such as Deleware

            \item Avoid raising questions of successorship unless the students raise them; see Baltimore City case study regarding five at-large members
        \end{itemize}
    \end{itemize}

    \item Alternate construction:  Baltimore City Government
    \begin{itemize}
        \item Current City government is elected strong-mayor with a unicameral legislature, 14 wards of 45,000 population plus one at-large City Council President, and four-year terms for all

        \item Proposition:  ten wards, five at-large by STV, City Council President selected by City Council from its five at-large members
        \begin{itemize}
            \item Ward size is comparable to Baltimore County

            \item Contrast five at-large by STV with Condorcet in 14 wards and a Condorcet Council President

            \item Contrast 15 wards with a supermajority- or Condorcet-elected Council President elected from and by the Council body with the proposed five at-large and Council President elected exclusively from those

            \item Consider backfilling a vacancy in the five at-large members.  There are several common methods
            \begin{itemize}
                \item Callback:  protect all remaining members as elected, withdraw the vacating member, and recompute the STV election without them (this may be more sensible with Geller-STV, but you have to automatically assume quota if you're recalculating Borda scores from fractional vote transfers).  Note STV can change all members, and so just rerunning it can return a bunch of candidates who may not represent those voters whose candidate vacated, hence the protection of these candidates.

                \item Special election:  Can only really be Condorcet.

                \item Successorship:  Vacating member nominates a successor; or multiple successors for either a special election or appointment by the body of City Council, or by some kind of Appointment Committee.  Informally request the member nominate at least one viable callback candidate.

                \item Appointment:  Successorship but the member doesn't have any formal privilege in determining who is appointed.

                \item Stall:  if close to the end of the session or the end of the term, do nothing.  Use two-year terms for the at-large members.  May combine well with callback or successorship.
            \end{itemize}
        \end{itemize}

        \item Proposition:  Council-manager structure in the 10/5 proposition above
        \begin{itemize}
            \item Democratic legitimacy of City manager appointed by City Council
            \begin{itemize}
                \item City Council is elected by voters, and the voters must also approve a Charter giving Council the power to appoint the Executive

                \item City Manager may be more aligned to Council and so better execute as part of a functional government; but see the two-tier problem
            \end{itemize}

            \item Does the voting rule Council uses matter?  Is Condorcet necessary, or only supermajority?
            \begin{itemize}
                \item A Condorcet rule is complicated, but creates an all-majorities intersection protecting against $k=50$ simple majority rule

                \item A supermajority rule pays some regard to the two-tier a-priori voting power problem, but makes it more difficult to \textit{remove} the City Manager

                \item Caveats on removal:  in theory, only a few members need to switch positions to remove the City Manager—more with $k>50$.  In practice, people become less apt to change the outcome once it is in place for a little while.
            \end{itemize}

            \item ``Power of corruption''?
            \begin{itemize}
                \item An elected Executive can act more freely, without accountability to City Council, and so cannot be removed for corruption if the evidence doesn't reach a legal threshold

                \item An appointed Executive can be removed by a vote of Council members, and so the ``power of corruption'' is distributed such that Council members must be colluding \textit{or} the corrupt behavior must create no suspicion

                \item Corruption becomes harder to hide the more of it an official carries out, but this becomes more obvious to Council than to voters (compare recall elections of an elected mayor and debate whether Council would respond \textit{correctly} or at least to little harm, \textit{more reliably}, and \textit{more quickly} to corruption)

                \item Corruption is harder to hide \textit{when there are more co-conspirators}
            \end{itemize}
        \end{itemize}
    \end{itemize}
\end{itemize}

\begin{boxcomment}
    I examine STV vacancies in the City government model because the five elected candidates are more diverse than when electing three.  This is an interesting consideration:  when electing more winners, appointing or electing at-large to fill a vacant seat disenfranchises fewer voters (by selecting a candidate not well-aligned to them), but excludes their voice to a much greater degree than when electing fewer winners.  At the same time, other winning candidates are ideologically closer, diminishing the degree of exclusion, but not eliminating it.

    With an odd number of candidates, one should be approximately the Condorcet candidate (STV-B might make this highly-likely, but can't identify which is the Condorcet candidate); in any case, candidates become more dissimilar in pairs, as per Plott's Majority Rule Equilibrium Theorem and Plott symmetry.  Replacing the furthest-out candidates with a Condorcet winner has obvious implications.
\end{boxcomment}