The syllabus is adjusted to fit the new layout of the curriculum.  Of a 15-week session, 14 lessons are taught, with one vacation day.

{
    \newcommand{\sylweek}[3]{

%        Week \ref{cur:#1} & \textbf{\hyperref[cur:#1]{#2}} \\
        Week \ref{cur:#1} & \textbf{#2} \\
        \hline
        \multicolumn{2}{|X|}{\centering
            #3
        } \\
        \hline
    }
    \begin{table}[h]
        \label{syllabus}
        \centering
        \begin{tabularx}{\linewidth}{|c|c|}
            \hline

            \sylweek{what-is-sc}{What is Social Choice?}{What is choice?  What is social choice?  Why is it relevant today?}

            \sylweek{ed-history}{Education, Knowledge, and a History of Social Choice}{Covers the role of knowledge and education in democratic governance.  Reviews the history of social choice thery.}

            \sylweek{collective-decisions-consent}{Ethics, Welfare, and Consent}{Covers basics of policy and welfare ethics and consent to  Constitutional rules.}

            \sylweek{public-discourse}{Public Discourse and Good Collective Decisions}{The role of public discourse in democracy, theories of why and how the public reason trends toward optimum collective decisions, and the aspects of good collective decisions.}

            \sylweek{majority}{Majority Rule and Preferential Voting}{Covers majority rule, preferential voting, tournament sets, and the democratic legitimacy implications thereof.}

%            \sylweek{ethics}{Empty}{Ethics and Welfare was merged into other weeks.}

            \sylweek{representative-democracy}{Representative Democracy}{Discusses the purpose and legitimacy of elected representation.}

            \sylweek{single-winner-voting-rules}{Single-Winner Voting Rules and Elections}{Covers technical voting rule criterion for single-winner rules, and explores election structures.}

            \sylweek{multi-winner-voting-rules}{Multi-Winner Voting Rules and Elections}{Covers multiple-winner rules, and further explores election structures.}

            \sylweek{government-structure}{Government Structure}{Explores the structure of government and the interaction with social choice.}

            \sylweek{evaluating-decision-methods}{Evaluating Decision Methods}{Explores decision methods in terms of efficiency, outcomes, equity, and other measures.}

            \sylweek{manipulation}{Electoral Manipulation}{Demonstrates and discusses manipulating outcomes by tactical voting or strategic nomination, the degree to which various voting rules resist manipulation, and historical examples of political parties and governments manipulating elections.}

            \sylweek{propaganda}{Investigating Propaganda}{A lesson debating and discussing propaganda from electoral reform advocates and opponents.}
        \end{tabularx}
    \end{table}
}

The curriculum aims to provide a baseline understanding of social choice and voting theory.  This gives students a starting point in understanding various functions and principles underlying democratic government, rather than vague ideals such as ``majority rule'' or ``the right to vote.''  It is not a technical skills course; however, it forms a foundation of understanding upon which a technical skills course can build, and with which a student can analyze policy problems whether for ethics and legitimacy concerns or for purposes of implementing more representative elections and government institutions.

As a college curriculum, this course uses classroom discussion with minimal evaluation.  Instructors should encourage participation from students who engage less than their peers, and focus primarily on engagement with the discussion and the material.  A non-technical term paper can demonstrate student engagement, showing the student gained knowledge and insight from the class, and should be of a form not focused on reciting facts or engineering a specific expected solution to a problem.

In this way, this course is more similar to one covering critical thinking or theories of justice, rather than one covering history or mathematics.

\textbf{General notes}

Tideman's taxonomy may fit into Public Discourse and Good Collective Decisions, but this seems wedged in there.  Perhaps move ``Evaluating Decision Methods'' up?

The empty spot from merging Ethics and Welfare into Consent might be a good place for Chapter 11 spatial model; or into Government Structure.



