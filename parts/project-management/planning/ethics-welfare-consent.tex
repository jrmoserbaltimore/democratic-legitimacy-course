\chapter{Ethics, Welfare, and Consent}

This chapter incorporates Ethics and Welfare from Week 6 into the Constitutional Consent from week 4.  This is a more coherent framework, and sets up for epistemic democracy concepts.  Rawls's public reason fits into this framework, as the public reason seaches for the overall knowledge of the collectivity, and thus can be said to be seeking a decision from behind the veil.  Black's median voter theorem and Plott's majority rule equilibrium theorem also build on Rawls in this way.

The major flow is as follows:

\begin{itemize}
    \item Explain Pareto welfare ethics.

    \item Discuss the Income Effect and tie to Pareto and the inability to truly compare welfare between individuals.

    \item Discuss Kaldor-Hicks, showing the income effect leaves a gap because we can't transfer money and then transfer it back without ending where we started.  (Scitovsky's Double addresses this)

    \item Buchanan-Tullock, discussing ``overall Pareto improvement'' and consent to a society that's better in the long run for everyone.

    \item Rawls's veil of ignorance and consent behind the veil.
\end{itemize}

The taxonomy of decisions is set aside for this lesson, at the moment; will need to explore if it fits here properly.

Welfare-based voting systems are not described here.  Perhaps move these to single-winner voting systems?

Transitive and intransitive preferences become part of the Majority Rule and Preferential Voting lesson, where the Condorcet criterion and the Smith Set are discussed.

\section{Assigned Reading}

\begin{itemize}
    \item Chapter 3 of the Handbook of Social Choice and Voting \autocite[35-51]{Heckelman2015} covering Constitutional consent (Buchanan-Tullock, Rawls)

    \item A custom reading covering the Income Effect, Pareto, Kaldor-Hicks, and Scitovsky's Double; reference Collective Decisions and Voting \autocite[23-32]{Tideman2006}

    \item Is Everything Really Up for Grabs? \autocite{Kassner2006}

    \item Reflections on the Enslavement of Negroes \autocite{Condorcet1781}

    \item On the Admission of Women to the Rights of Citizenship \autocite{Condorcet1789}
\end{itemize}

\section{Research and Content Notes}

Consider \autocite{Kassner2006} in relation to \autocite[45]{Heckelman2015}: ``a policy that was produced by a legitimate process is a legitimate policy that is founded on the agreement of the citizens.''  Both Rawls and Kassner argue that certain policies are not legitimate regardless of any process.

Kassner's argument that democracy is a set of principles suggests the contractarian view is an obligation:  if government is by consent, then the people must recognize when the process is insufficient for the people to actually hold public officials accountable.  This will be explored in proportional representation systems, notably list proportional versus single transferable vote.

