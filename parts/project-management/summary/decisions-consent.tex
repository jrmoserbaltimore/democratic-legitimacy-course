\chapter{Collective Decisions and Consent}
This is being merged into an ``Ethics, Welfare, and Consent'' lesson.

\section{Summary}

As-is, this lesson covers Tideman's taxonomy, Buchanan-Tullock constitutional consent, and Rawls's veil of ignorance theory.

Tideman's taxonomy should be ejected from this and moved to another lesson; the complication of how to start a democratic government, considreing you need unanimous consent, should be addressed by Buchanan-Tullock and Rawls.

\subsection{Content}

\subsection{Annotations}

Is Rawls's concept of a democracy as a joint venture important here? [Probably not; that's getting into too much depth]

I considered leaving Condorcet's writings on the rights of women for the next class to facilitate debate in this session, as the arguments can be used to challenge students for a response; however, the arguments are of a nature as to be impossible to naturally reason out without the devotion of time, study, and deep thought on the matter.

This is a good place to ask whether democracy is a process or a set of principles, as per Kassner; consent implies principles, not process.

This ties together public discourse, Condorcet jury theorem, and consent, where a democracy tries to bring this all together to discover the veil-of-ignorance perspective. [\textbf{Push this to a later lesson}]

\section{Sources}

\begin{itemize}
    \item Chapter 3 of the Handbook of Social Choice and Voting \autocite[35-51]{Heckelman2015}
    \item Is Everything Really Up for Grabs? \autocite{Kassner2006}
    \item Appendix 2, modes of collective decision making \textbf{[remove this]}
    \item Condorcet's reflections on the enslavement of negroes \autocite{Condorcet1781}
    \item ``On the Admission of Women to the Rights of Citizenship'' \autocite{Condorcet1789}
\end{itemize}

\section{Concepts}

\begin{itemize}
    \item Collective decision modes [\textbf{Move this somewhere else}]
    \item Constitutional consent (Buchanan-Tullock, Rawls veil of ignorance)
    \item Legitimacy of democratic governance with excluded classes unable to vote or participate in general
\end{itemize}

\section{Missing Information}

