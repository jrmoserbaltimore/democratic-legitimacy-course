\chapter{Public Reason and Good Collective Decisions}

\section{Summary}

This lesson introduces epistemic democracy concepts, including Rawls's public reason and the Condorcet jury theorem.  Formalization of the concept of knowledge is delayed until this lesson, to improve discussion in the prior lesson.

\subsection{Content}

\begin{itemize}
    \item Knowledge and jury theorem as a basis for epistemic democracy.
    \begin{itemize}
        \item How do we know who actually knows things?
        \item Experts disagree on significant facts.
        \item Voters disagree with each other based on experiences.
    \end{itemize}

    \item Rawls's public reason.
    \begin{itemize}
        \item The purpose of the public reason relating to knowledge and jury theorem.
        \item The actors in public discourse shaping the public reason, including political campaigns, media, educators, neighbors, etc..
        \item What is not part of the discourse (fundamental rights are immutable).
    \end{itemize}

    \item Voting and human rights in terms of epistemics
    \begin{itemize}
        \item Black slavery, women's suffrage, black codes, felony disenfranchisement
        \item Foreign immigrants and voting rights
    \end{itemize}

    \item Epistemics of non-voting
    \begin{itemize}
        \item The ice cream example (when equal preference is correctly a non-vote and why)
        \item Non-voting and perceived pivot probability (don't vote because it doesn't matter—this is a problem, as it makes outcomes of elections worse)
    \end{itemize}
\end{itemize}

\subsection{Annotations}

FOr immigration, consider Virginia and Kentucky papers, and Hamilton's response in The Examination.

This lesson needs to address that people have knowledge of their life situation, and will exchange knowledge through public discourse.  Political campaigns and advocacy groups share part of this.  Bayes's theorem is illustrative here; the public discourse applies new information to Bayes's theorem.

One annotation discusses mandatory voting and straight-party ticket, mostly seen here already.

\section{Sources}

\begin{itemize}
    \item \autocite[Chapter 9]{Heckelman2015}
    \item \autocite{Rawls1997}
    \item \autocite{Estlund2008}
    \item \autocite{Swenson2007}
\end{itemize}

\section{Concepts}

\begin{itemize}
    \item Epistemic democracy
\end{itemize}

\section{Missing Information}

What actually makes a ``good collective decision.''

Tideman describes this in \autocite[69-72]{Tideman2006} by the following criteria:

\begin{enumerate}
    \item The decision is made by a procedure that is accepted by the members of the collectivity.
    \item The outcome does not conflict with the reasonable claims of persons who are not members of the collectivity.
    \item Any member of the collectivity who disagrees with the outcome can leave on reasonable terms.
\end{enumerate}

These are not possible to fully establish, but are the target conditions.

Use this to rewrite the reading on knowledge and good collective decisions.
