\chapter{Education, Knowledge, and History of Social Choice}

\section{Summary}

This lesson introduces some of the history of social choice.  It has three main goals:  create historical context; lay the foundation for democratic legitimacy, with an exploration of what it means to ``know'' something; and show the repeated loss of social choice theory in the context of education and Yastrebtseva's ``tyranny feeding on popular ignorance.''

\subsection{Content}

\begin{itemize}
    \item A review of social choice theory history as in chapter 2 of the text \autocite{Heckelman2015}.
    \item Basic concepts of the importance of education
    \begin{itemize}
        \item Driven mainly by Yastrebtseva's paper

        \item Condorcet's assertion of education inequality being a main source of tyranny, and Yastrebtseva's caution of society without education being tyranny feeding on ignorance

        \item Condorcet's ``political religion''

        \item Condorcet's writings and work against slavery

        \item Condorcet's call for a unicameral legislature
    \end{itemize}

    \item A discussion of knowledge
    \begin{itemize}
        \item The three criteria describing what it means to ``know''

        \item Discussions of what we may and may not know, particularly of whether elections are democratic
    \end{itemize}
\end{itemize}

\subsection{Annotations}

\begin{itemize}
    \item This should lead to a facilitated discussion about education and what it means to ``know''

    \item Contrast ``political religion'' of centralized education with local decentralized education, mainly potentially tyranny by central control of the public mind versus difficulty providing universal high-quality education in a decentralized system

    \item Students come to understand the importance of education to democracy; education inequality as tyranny

    \item Slavery becomes highly relevant to democratic legitimacy due to exclusion from public reason and discourse, immutability of human rights, and consent behind the veil of ignorance to a society allowing slavery
\end{itemize}

\section{Sources}

\begin{itemize}
    \item \autocite[Chapter 2]{Heckelman2015}
    \item \autocite{Yastrebtseva2015}
\end{itemize}

\section{Concepts}

\begin{itemize}
    \item Historical figures in social choice

    \item Philosophical basis of what knowledge is

    \item Education as a tool of control and a tool of freedom
\end{itemize}

\section{Missing Information}

\begin{itemize}
    \item Does not provide a concrete explanation of what to cover from history

    \item Discusses Condorcet's ``political religion'' without comparing late-19th-century France moving to universal secular education for both men and women (assigned reading by Yastrebtseva covers this)

    \item An explicit but limited discussion of Condorcet's opposition to slavery and the death penalty, and his calls for universal human rights

    \item Sort of bypasses Borda; should cover that they had debate over fundamental approaches to voting

    \item Should  include specific arguments about knowing things
\end{itemize}

Unicameral/Bicameral legislature should be pushed back to the discussion on government structures, fitted into the spatial model of American institutions.

