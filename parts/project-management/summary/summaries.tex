\part{Review}

\chapter{What is Social Choice}

\section{Summary}

\subsection{Content}

\begin{itemize}
    \item Concepts
    \begin{itemize}
        \item \textbf{What is Democracy?}

        Covers normative definition of democracy.  References \autocite{Kassner2006}.  Assumes democracy is a principle of self-rule based on popular sovereignty.

        \item \textbf{What is a majority?}

        Discusses ``most votes wins'', plurality vs. majority, IRV vs. Condorcet.

        \item \textbf{When is voting democratic?}

        Discussion of who should and shouldn't win, given the voting demonstration.  Do students find some of these systems more or less democratic?
    \end{itemize}

    \item Examples and Demonstrations
    \begin{itemize}
        \item \textbf{Election demonstration}

        Demonstrating Plurality, IRV, and Condorcet by an example election.  A brief breakdown of a three-way election into pairwise elections is shown.

        \item \textbf{Two-tier example}

        100 districts, 100 senators, 100 voters, showing 26\% of voters controlling the Senate.
    \end{itemize}
\end{itemize}

The lesson text explains the course's hostile debate technique, wherein the instructor intentionally leads students into discussions drawing on expected incomplete or incorrect knowledge.  This draws students to commit firmly to their existing beliefs, and the sudden conflict with their expectations fixes the new concepts within their minds.

Particular statements and concepts to draw out of students and then undermine:

\begin{itemize}
    \item Most votes wins

    \item ``The majority'' (versus the existence of many majorities) or ``a majority'' (with the apparent assumption that this is sufficient)

    \item Suggesting to replace plurality with a runoff-type election

    \item City Council or Senate voting by a majority of members, before laying out the two-tier problem. (This will be expanded on later, with the complexity added by bicameral legislatures providing the protection of supermajority, versus the Senate's unilateral actions e.g. approving appointments)
\end{itemize}

\subsection{Annotations}

\begin{itemize}
    \item Sometimes a savvy student enrolls.  Prevent them from getting ahead of the students; pull them aside and suggest they wait for people to make some mistake before volunteering some interpretation, ``letting the line out slowly.''

    \item Some apparently savvy students are only familiar with advocacy and not theory, e.g. those calling IRV ``ranked choice voting.''

    \item If students suggest Chris didn't get a majority in the example election, coax them into suggesting a runoff.  If they suggest a tournament, lead the discussion to multiple majorities from there; otherwise only show that two candidates have a majority, but don't discuss the concept much.

    \item Avoid discussing direct vs. representative democracy.

    \item Some of the written material is getting deep into certain topics, e.g. ``what is a majority?''

    \item Timing of party primary discussion needs to be determined.  (This is now somewhere around single- and multi-winner rules and elections built on them; the annotation wasn't updated)

    \item May want to note to students that the course was originally intended to be a technical study with some philosophy along the way, but eventually shaped into a course working through the philosophy of democracy and building the technical discussion on top.

    \item Make sure the class ends with hammering in that this course exists to pass on the knowledge required for a people to defend democratic governance from bad policy and undemocratic elections.
\end{itemize}

\section{Sources}

\begin{itemize}
    \item \autocite{Kassner2006}
\end{itemize}

\section{Concepts}

A brief overview of several concepts:

\begin{itemize}
    \item Democracy is a set of normative principles to which a process is bent, not a process with no principles

    \item Majority rule, ``what is a majority?''

    \item ``When is voting democratic?''

    \item Different types of voting rules, tangentially, when showing plurality, runoff, and Condorcet outcomes

    \item Barely touch representative democracy with the two-tier example
\end{itemize}

\section{Missing Information}

Should attempt to draw out something about ``more democracy'' as if this actually means something, but avoid going into much detail here.  Particularly, avoid discussions of supermajorities as ``less democratic'' or ``minority rule'' in favor of blindsiding students in the lesson covering supermajorities.

There's nothing in here addressing IRV's specific faults, which becomes important if students are familiar with IRV advocacy.  Particular considerations:

\begin{itemize}
    \item For close elections, IRV completely excludes the voters outside the mutual majority

    \item This exclusion requires voters to consider whether they should vote for the opposition as a safety measure (use a 2016 Presidential election example)

    \item This situation is analogous to party primary elections, wherein a voter must decide in which party primary to vote, ensuring that voters essentially vote for party, and only the majority party's primary voters determine candidate (reverse the election logic).
\end{itemize}