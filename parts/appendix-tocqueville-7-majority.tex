\chapter{Of the Omnipotence of the Majority in the United States and its Effects}

\nocite{Tocqueville2010}

%Democracy in America: Historical-Critical Edition of De la démocratie en Amérique, ed. Eduardo Nolla, translated from the French by James T. Schleifer. A Bilingual French-English editions, (Indianapolis: Liberty Fund, 2010). Vol. 2.

The very essence of democratic governments is that the dominion of the majority be absolute; for, in democracies, nothing outside of the majority can offer resistance.

Most of the American constitutions have also sought to augment this natural strength of the majority artificially.

Of all political powers, the legislature is the one that most willingly obeys the majority. The Americans have wanted the members of the legislature to be named directly by the people, and for a very short term, in order to force them to submit not only to the general views, but also to the daily passions of their constituents.

They have taken the members of the two houses from the same classes and named them in the same way; in this way, the movements of the legislative body are almost as rapid and no less irresistible than those of a single assembly.

Within the legislature thus constituted, the Americans gathered together nearly the entire government.

At the same time that the law increased the strength of powers that were naturally strong, it weakened more and more those that were naturally weak. It gave to the representatives of the executive power neither stability nor independence; and, by subjecting them completely to the caprices of the legislature, it took from them the little influence that the nature of democratic government would have allowed them to exercise.

In several states, the law delivered the judicial power to election by the majority; and in all, it made the existence of the judicial power dependent, in a way, on the legislative power, by leaving to the representatives the right to fix the salaries of judges annually.

Customs have gone still further than the laws.

In the United States, a custom is spreading more and more that will end by making the guarantees of representative government empty; it happens very frequently that the voters, while naming a deputy, trace a plan of conduct for him and impose on him a certain number of definite obligations from which he cannot deviate in any way. Except for the tumult, it is as if the majority itself deliberated in the public square.

Several particular circumstances in America also tend to make the power of the majority not only predominant, but irresistible.

The moral dominion of the majority is based in part on the idea that there is more enlightenment and wisdom in many men combined than in one man alone, more in the number than in the choice of legislators. It is the theory of equality applied to minds. This doctrine attacks the pride of man in its last refuge. Consequently the minority admits it with difficulty and gets used to it only with time. Like all powers, and perhaps more than any other, the power of the majority thus needs to last in order to seem legitimate. When it is beginning to be established, it makes itself obeyed by force; only after living under its laws for a long time do you begin to respect it.

The idea that the right to govern society belongs to the majority because of its enlightenment was carried to the soil of the United States by the first inhabitants. This idea, which alone would be enough to create a free people, has today passed into the mores, and you find it in the least habits of life.

The French, under the old monarchy, held as a given that the king could do no wrong; and when he happened to do something wrong, they thought that the fault was with his advisors. This facilitated obedience marvelously. You could murmur against the law, without ceasing to love and respect the law-maker. Americans have the same opinion about the majority.

The moral dominion of the majority is based as well on the principle that the interests of the greatest number must be preferred to those of the few. Now, it is easily understood that the respect professed for this right of the greatest number naturally increases or decreases depending on the state of the parties. When a nation is divided among several great irreconcilable interests, the privilege of the majority is often unrecognized, because it becomes too painful to submit to it.

If a class of citizens existed in America that the legislator worked to strip of certain exclusive advantages, held for centuries, and that he wanted to bring down from an elevated position and restore to the ranks of the multitude, it is probable that the minority would not easily submit to his laws.

But since the United States was populated by men equal to each other, no natural and permanent dissidence is yet found among the interests of the various inhabitants.

There is such a social state in which the members of the minority cannot hope to attract the majority because to do so it would be necessary to abandon the very object of the struggle that the minority wages against the majority. An aristocracy, for example, cannot become a majority while preserving its exclusive privileges, and it cannot allow its privileges to slip away without ceasing to be an aristocracy. [In these countries, it is almost impossible for the moral power of the majority ever to succeed in being recognized by all.]

In the United States, political questions cannot be posed in as general and absolute a way, and all parties are ready to recognize the rights of the majority, because all hope one day to be able to exercise those rights to their profit.

So in the United States the majority has an immense power in fact and a power of opinion almost as great; and once the majority has formed on a question, there is, so to speak, no obstacle that can, I will not say stop, but even slow its course and leave time for the majority to hear the cries of those whom it crushes as it goes.

The consequences of this state of affairs are harmful and dangerous for the future.