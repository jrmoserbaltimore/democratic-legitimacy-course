
\refstepcounter{syllabuslesson}
\label{cur:public-discourse}
\section{Public Discourse and Good Collective Decisions\hyperref[syllabus]{↑}}

This lesson continues from Week 2, bridging individual knowledge with group knowledge.  The public discourse is the development of group knowledge, and the Condorcet Jury Theorem and its derivatives explain the epistemic means by which democracy leverages the knowledge of the collectivity.

Public discourse is the cure to uninformed voters.  This does not necessarily change their vote:  they may discover they are correct, but for the wrong reasons.

The rational non-voter will come up later when discussing representativeness of an election and its sensitivity to small groups of voters.  Certain electoral processes increase pivot probability, which increases the utility of voting if individuals believe their society's social compact influences other voters to vote.  In other words:  a culture of democratic ideals gives people reason to believe that they should vote because others must feel they should also vote, and because they become a part of the social example strengthening this commitment which drives others to vote, thus their individual vote may not matter, but the fact that they do vote matters a great deal.


\subsection{Assigned Reading}

Relevant reading assigned prior to this day:

\begin{itemize}
    \item Chapter 9 of the Handbook of Social Choice and Voting \autocite[140-158]{Heckelman2015}

    \item \prettyref{apx:on-knowledge}, covering knowledge and what makes a good collective decision.  The nature of collective knowledge only rises naturally from extremely long debates and broad study.

    \item Rawls on public discourse \autocite{Rawls1997}

    \item Estlund on epistemic democracy \autocite{Estlund2008}

    \item A rationale for abolishing compulsory voting in Australia \autocite{Swenson2007}

\end{itemize}

\autocite{Swenson2007} gives good arguments against compulsory voting.  The examination of the public discourse necessarily leans against compulsory voting because disinterested non-voters have no preference, and therefor correctly express their preference by not voting.  Forcing these voters to make a choice actively undermines the public discourse.

In Australia, and in jurisdictions where ballots must rank all candidates to be legitimate, donkey voting is common—which also produces votes not expressing the voter's preferences, but rather enforced at the expense of legitimate, correct preferential expression.  Donkey voting is recognized universally as a problem; bullet voting—ranking only one candidate—is also considered a problem.  Both of these are often argued from the position that the voter should rank more or all candidates, and should rank them honestly—ignoring that all unranked candidates are equally ranked, and least-preferred candidates for which a voter is indifferent are correctly ranked by not ranking them.  If anything, the inability to express equal rankings above the last—which is also not voting in certain pairwise races—is a problem.

\subsection{Overview}

\begin{itemize}
    \item Briefly revisit the concept of knowing, and discuss the differences in people's knowledge

    \item Move the discussion to group decisions and the difficulty in identifying who knows the correct action.  We have all kinds of tools for this in real life:  Ph.D.s, professional experience, social recognition of personal hobbies and life experiences, and yet credentialed experts disagree about significant facts all the time

    \item Lead the discussion to debate, campaigns, advocacy groups, and other forms of public discourse.  The classroom discussion, and education in general, is one instance of such discourse

    \item Epistemic democracy and the Condorcet jury theorem are the natural topic of focus from this point
    \begin{itemize}
        \item Questions about knowledge versus social factors, relating to the public discourse's tendency to be driven by supposed experts and whether these are merely a type of celebrity, are appropriate

        \item If the consideration is reached that experts are educating the public and thus we would be better off if they became the aristocracy, consider that these experts often disagree, and that different voters regard them differently, thus are evaluating the experts based on their own individual knowledge, and may frequently replace them via elections
    \end{itemize}

    \item Bring the discussion to human rights, including women's rights, slavery, black codes, and felony disenfranchisement, in light of the public discourse and epistemic democracy
    \begin{itemize}
        \item There is common concern that foreign immigrants will gain citizenship and pollute American culture and ideals as they gain too much population share and have too much influence on American elections.
    \end{itemize}

    \item Raise the question of non-voters
    \begin{itemize}
        \item Uninformed voters \textit{are} voters:  collective decisions are not made by elites deciding who has the right opinions

        \item Disinterested voters are \textit{not} voters:  a member of a collectivity who has no interest in a particular decision has equal preference for all decisions and judges all to be of equal welfare for themselves.  The correct vote is no vote at all, as this perfectly expresses their will.
        \begin{itemize}
            \item When disinterested voters are forced to vote, they are required to cast votes against their preferences

            \item When a vote between two options $\left\{A,B\right\}$ selects $A$, forcing abstaining voters to vote may change the result to $B$.  Disinterested voters favor $A$ and $B$ equally; if they vote $B$ and change the result, they are neither more nor less satisfied, but a larger number of voters are ultimately dissatisfied as a result.
        \end{itemize}

        \item Rational non-voters are \textit{not} implicitly non-voters:  the paradox of voting is that more people vote than we'd expect.  The likelihood of a single vote changing the outcome is low, so there is no utility to any individual voting, but a lot of cost.  People tend to vote when they believe pivot probability is high, and otherwise by social normative values.  Democracy mainly relies on individual voters knowing they have no power of enforcement over others, but nevertheless taking that step themselves in hopes that others will stand with them and come out to vote.
    \end{itemize}
\end{itemize}

\begin{todo}
    For immigration, consider the Virginia and Kentucky papers, and Hamilton's responses.  This may be an appropriate reading for the high school curriculum.
\end{todo}

\begin{todo}
    This lesson needs to address that people have knowledge of their life situation, and will exchange knowledge through public discourse.  We see this in political campaigns and advocacy groups appealing to the general public.  The Condorcet jury theorem and its derivatives come up here.

    Bayes's theorem proves illustrative here, as knowledge is affected by new information.  The public discourse, in seeking consensus by transfering knowledge, applies this new information within Bayes's framework.
\end{todo}

\begin{todo}
    There's a huge debate over mandatory voting and straight-party ticket.  Group voting ticket is too horrific to talk about much.  Mandatory voting and straight-party ticket encourage non-voters to vote, in that a voter naturally not voting in a particular election is expressing indifference—agreement with whatever decision the group would make without them.  The difference between rational non-voters and disinterested voters is that rational non-voters have an opinion, but don't believe their vote will change the outcome; disinterested voters cannot cast any correct vote except no vote.  The public discourse would in theory turn disinterested voters into rational [non-]voters.

    Maybe save this for discussions of election processes.
\end{todo}

\subsection{Traps}

\begin{itemize}
    \item Felony disenfranchisement leads to people suggesting everyone should be able to vote ``after they served their time'' or that they shouldn't be able to vote ``because they broke society's rules.''  Either suggests a period when someone's vote is invalid; the latter especially, which suggests we know the laws are just and correct, and that people who break them are wrong and so removing them from the public discourse improves our collective knowledge—an untenable position.  Bring up marijuana legalization in the middle of this conversation, if you can goad the students into making that assertion; or otherwise, where it seems appropriate.

    \item A discussion on mandatory voting has enough pitfalls to yank students around a bit.  People won't fall for the requirement to rank all 30 or 40 candidates; they \textit{will} fall for the idea that voting is always more correct than not voting, due to a blind spot for the proposition that somebody may see all options as exactly equal and thus casting any vote misrepresents their preferences.
\end{itemize}

\subsection{Assessment}

\subsubsection{Multiple Choice}

\begin{itemize}
    \item \multiplechoice{By what manner did Rawls suggest constitutional democratic societies make and accept decisions despite irreconcilable differences of ideals?}{Public discourse}{Civics education}{Religion}{Representative democratic government}
\end{itemize}