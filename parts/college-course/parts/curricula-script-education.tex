\refstepcounter{syllabuslesson}
\label{cur:ed-history}
\section{Education and History of Social Choice\hyperref[syllabus]{↑}}

This lesson briefly touches on the disagreements between Condorcet and Borda.  Importantly, Condorcet's work was shelved for 190 years, until Black in the late 1940s found the only copy of Condorcet's works in an Oxford library, published in 1784, unopened, never read.  Two centuries of ignorance.

This prepares for Week 3, which covers collective decisions and public discourse.  Both weeks are sequenced intentionally before ethics, preference, and welfare.  The debate between preference and welfare isn't just about which system is better or whether welfare-based systems fail due to unethical voter behavior, but also about whether preference systems are also welfare systems.

\subsection{Assigned Reading}

Relevant reading assigned prior to this day:

\begin{itemize}
    \item Chapter 2 of the Handbook of Social Choice and Voting \autocite[15-31]{Heckelman2015}

    \item Yastrebtseva covering Condorcet's doctrine of public education \autocite{Yastrebtseva2015}
\end{itemize}

Reading assigned for next week:

\begin{itemize}
    \item \prettyref{apx:on-knowledge}, covering knowledge and what makes a good collective decision.  This is assigned only for the next day so the class can debate over what knowledge \textit{is} rather than recite what they've read.  It fits there, too.
\end{itemize}

\subsection{Overview}

\begin{boxcomment}
    Tis class leads into a facilitated discussion exploring education and digging out what it might mean to ``know'' something.  Condorcet's ``political religion'' contrasts with the more localized problems in decentralized education.  Students should come to understand the importance of education to democracy; education inequality as a source of tyranny; and the difficulties inherent in providing universal high-quality education while protecting the education system from assault by tyrants seeking to subjugate the public mind.
\end{boxcomment}

\begin{itemize}
    \item A small bit of the history of social choice theory
    \begin{itemize}
        \item The important point to cover is that nobody read Condorcet's work, and the whole field was stunted for two centuries

        \item Covering the major names—Condorcet, Dodgson, Black, Arrow—is interesting, especially with Black and Arrow finding out Condorcet had beat them to their major discoveries by 200 years, but they didn't know because nobody in the world even read his 1790 essays until 1950

        \item Condorcet was one of the original founders of the Society of the Friends of Negroes, and got slavery banned in the French constitution of 1792, but Napoleon opened the slave trade again in 1802

        \item Condorcet argued that a unicameral legislature was just as good as bicameral; bicameral is superior, but that's for government structure
    \end{itemize}

    \item Without education, we only have tyranny feeding on ignorance; this course addresses that for whether our elections are democratic
    \begin{itemize}
        \item This is a good time to discuss Condorcet's writings on education, explored by \autocite{Yastrebtseva2015}

        \item Centralized education leads to a ``political religion'' where the central authority decides what people will think, which is dangerous

        \item Decentralized education can lead to inequalities in education, which can only lead to tyranny

        \item Education, as a social institution, is extremely difficult, dangerous, and important
    \end{itemize}

    \item \textit{What does it even mean to know something?!}
    \begin{itemize}
        \item See \prettyref{apx:on-knowledge} and \autocite[57-63]{Tideman2006} for the subject matter

        \item Lead the discussion as per the subject described in these resources

        \item Try interesting arguments, e.g. that slavery never existed and cannot exist; rather there is only tyranny by which the mere existence of a law is a violent crime against those who may come to be viewed by the law as property
    \end{itemize}
\end{itemize}


\begin{boxcomment}
    Slavery becomes highly relevant to democratic legitimacy.  Knowledge and public discourse indicate that democracy is non-functional if a significant portion of the population is excluded; human rights are immutable and universal; Rawls's veil of ignorance raises important considerations about whether anyone would consent to a society that even allows slavery.
\end{boxcomment}

\subsection{Traps}

There aren't any here.

\subsection{Assessment}

\subsubsection{Multiple Choice}

\begin{enumerate}
    \item \multiplechoice{Name one of the main sources of tyranny.}{Education inequality}{Taxes}{Unicameral legislatures}{Term limits}

    \item \multiplechoice{What concern did Condorcet raise against a centralized education system?}{It would become a political religion}{arg3}{arg4}{arg5}

    \item \multiplechoice{What principle underlies the expectation that democratic societies tend to produce the best outcomes for all members?}{Condorcet jury theorem}{Median voter theorem}{Equal representation}{The right to education}
\end{enumerate}