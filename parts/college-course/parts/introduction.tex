\chapter{Introduction}
This curriculum guides instructors through the 14 weeks of instruction necessary for a 15-week semester.  As a philosophy course giving students a way of thinking, it relies on facilitated\footnote{It may help to be familiar with dynamic facilitation.} in-class discussion guided toward the major insights and points of debate the students should reach if they don't move along themselves.

Instructors should familiarize themsleves with the material by reading the Handbook of Social Choice and Voting \autocite{Heckelman2015}; Collective Decisions and Voting \autocite{Tideman2006}; the papers by Rawls, Estlund, and Waldron \autocite{Rawls1997,Estlund2008,Waldron1998}; Kassner's response to Waldron \autocite{Kassner2006}; and the supplemental material presented here.  Instructors should be familiar enough with voting theory to be comfortable reading the paper covering Condorcet-Hare methods \autocite{GreenArmytage2011}, and should familiarize themselves with Pareto, Kaldor-Hicks, Buchanan-Tullock, and Rawls's veil of ignorance if these are new concepts.  The two writings by Condorcet should be studied vigorously \autocite{Condorcet1781,Condorcet1789} while applying these philosopihcal theories.

This course explicitly does not require students to memorize methods of spatial modeling or the mathematical definitions of voting rule criterion.  It aims to create a framework for thinking about democracy, spanning the epistemic justification, philosophical models of consent to democratic rule, and the vast technical considerations of implementation.  Students become aware of and can answer questions such as ``what is a majority?'' and ``when is voting democratic?''  Students explore voting rule criterion and vulnerability to various forms of manipulation to gain a broad sense of the implications of electoral processes.  Technical considerations for supermajority displace the automatic reflex to judge simple majority as the highest and most just form of democratic rule, instead providing the tools to analyze when simple majority is unable to satisfy the principles of democracy and when supermajority raises decision making costs while creating no benefit to democracy.

In short, this is not a course in applied technical concepts, but rather one to lay a framework.  Short essays and grading based largely on class participation are appropriate; things like weekly multiple choice quizzes are a formality at best.  A freeform essay asking students what value they believe this course provides is an appropriate course project.  Students may follow further, more technical study in public choice and voting theory; for those who don't, this course aims to push back against ``tyranny feeding on popular ignorance'' where elections are not as democratic as people believe \autocite{Yastrebtseva2015}.