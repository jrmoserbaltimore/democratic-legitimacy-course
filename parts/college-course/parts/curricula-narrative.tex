
\newcommand{\technical}{\textbf{[Technical]}}
\newcommand{\philosophical}{\textbf{[Philosophical]}}



Not entirely sure if 2-3 is going to be more a 1.5 or 1.25 session thing.  I'd like to fit chapter 2 of the Handbook on Social Choice and Voting into there, covering the history wherein Condorcet and Borda started on a bunch of stuff, then Condorcet's work was shelved for like 190 years, until Black in the late 1940s found the only copy of Condorcet's works in an Oxford library, published in 1784, unopened, never read.  I think it's a good pre-class reading that will lay a foundation for discussion on the need for education and the importance of knowledge.

This is going into Week 2 in part because I want to get the idea of public discourse sequenced before ethics, preference, and welfare; and because these things don't depend on that knowledge; and because the debate between preference and welfare isn't just about which system is better or whether welfare-based systems fail due to unethical voter behavior, but also about whether preference systems are also welfare systems (the Condorcet winner tends to be of high utility, close to the highest utility winner).  To really understand that, you have to understand public discourse as part of the democratic process; and to understand that, you have to understand the nature of knowledge and the difference between an individual's knowledge and the knowledge of the collectivity (i.e. Condorcet jury theorem and its implications).

Week 4 builds on this by discussing modes of collective decisions, and how we justify democracy and voting if we need unanimous consent.  There may be a lot of filler, but it's a good topic to fill…perhaps not that good, though, as this is a lot of space to fill.  Condorcet's writing pulls together the idea of consent and the public discourse and Condorcet jury theorem, as you can't possibly include the whole of the knowledge of a collectivity's members while excluding half the people in that collectivity.

Week 5 builds on weeks 2 and 3, as those two give a justification for the assertion that preferential systems—proper preferential systems that find the median voter—are high welfare.  It might be nonsense to assume they're not (consider the absurdity of weeks 2 and 3 coupled with the assertion that any individual or small group of elites can be identified who are able to determine who should be elected for the greatest good of all—the public choice theorists asserting that a certain candidate is in fact the one providing the greatest welfare, at least outside their models).

Week 6 brings together representative democracy, which touches on everything taught in this class so far.  I was sure Rawls had a theory of justice mentioning the equal right to hold office, but I'm less certain about anything that specifically claims representative democracy is good.  Touch on the two-tier problem here to keep open the question about representative democracy, what constitutes a majority, and when voting is democratic.

Week 7 enters a technical discussion about evaluating collective decision methods.  This builds directly on Tideman's taxonomy of decision making methods.  I doubt a lot will be taught here; there is a small bit of conceptual consideration which raises enormous, unending questions and the entire class will be high-quality debate.  That voting can cause inequality and unjust results due to the tyranny of the majority runs up against the Condorcet winner being selected by all majorities, and the median voter.  This is a good place to cite Jeremy Waldron about judicial review and ask questions about the authority of judges and executives, and even the authority of legislators—things automatically accepted in week 6.  I still want to examine this for more content and really need to dig into this week's lesson.

Week 8 gets into the more deep and dry technical discussions of voting methods and their evaluation specifically.  This is where we give students structure to analyze a voting method proposal.  We also introduce multi-winner districts and single transferable vote here.  Dr. Uzochukwu will likely appreciate the contrast between MNTV and STV; and Prof. Willis had brought up that multi-member districts were created as a tool to suppress black voters by multiplying the majority or even plurality vote for multiple candidates, something that can be replicated under single-member districts by careful district drawing but utterly fails under STV.  (We seriously need to get rid of MNTV, and Condorcet isn't good enough for this)

Week 9 brings this together with government structures.  I use this time to compare the outcomes for STV and MNTV in the House of Delegates, Plurality vs. Condorcet in the Senate or for a Governor, and the implications when using a parliamentary system appointing the executive.  I also use a modification of the structure used in the NCL's Model City Charter to demonstrate the careful crafting of a government as a whole (this is actually a system I proposed for Baltimore City).  At the State level, the demonstration shows a diversity of voices from each district in the House of Delegates, and a consensus of each district in the Senate, and in the Executive; at the City level, there are interesting interactions and implications for the Council body, and I tailored the specific details to distribute the "power of corruption" such that it takes an enormous amount of collusion to pull off what an elected Mayor can do in backroom deals, and such collusion is dangerous, hard to keep secret, and unstable across elections.  It's not just "we need our votes to count," but a whole examination of what the government structure accomplishes and how the method of voting and representation affects that.  This is also where I address K-majorities and the US Senate filibuster, and take more shots at Party Primaries and IRV to show how to make voting extremely undemocratic.

Week 10 takes up electoral manipulation.  This is where I weaponize the flaws in various systems.  Robbie Robinette gave a presentation on forcing IRV to fail by campaign strategy—something I'd weaponized by strategic nomination (running an additional candidate intended to lose so as to get rid of the winner).  Tactical voting I usually consider infeasible, but it has a place in score systems and in Borda; Group Voting Ticket—where a voter checks a party and deposits their ballot, and that number of ballots are given to the party to fill out as they see fit—is the most concentrated evil, giving essentially one voter (an administrative entity) an enormous amount of knowledge about many votes because that voter is allowed to cast them.  Where GVT or straight-party ticket is used, these options are selected as high as 90\% of the time, which underscores how dangerous and destructive these are.  It's disconnected from the rest, except insomuch as its logical relationship building on modes of collective decision and on the flaws that prevent voting from producing good outcomes and high equity, as well as to the concepts of representative democracy, government structure, and general democratic legitimacy which are predicated on voting actually being democratic.




There are some holes here and I need to fill a few more weeks of content.  I don't know what to use for that.  I don't want to go into a study of democratic history in earnest, the Federalist papers, the life and writings of Condorcet, or other filler material—I'd actually like to use all that for another class, possibly under HIST/PHIL, because I basically can't stand history and I have been consuming this history with such voracious appetite that I tried to learn French to get at Condorcet's original writings (I did learn Esperanto a couple years ago) and then obtained psychiatric care to deal with an array of issues preventing me from doing so (ADHD and cyclothymia are kind of terrible, but worse they were getting in my way).  If that class already exists where do I sign up?

I'll think of something.  There's the question of election security and a general exploration of electoral processes as a whole (primary to general election).  The method I have recommended is, happily, also now recommended by Nicolaus Tideman, although he has some additional modifications on STV.  Once I've taught the students how to think, they can withstand me suggesting what to think, or if not then I have failed to push back against tyranny feeding on popular ignorance.

From The Handbook of Social Choice and Voting, I have identified several relevant chapters to assign as readings (and I am creating several documents for readings):
Chapter 2:  History of social choice
Chapter 3:  Unanimous consent and constitutional economics
Chapter 4:  Rational choice and the calculus of voting
Chapter 6:  Majority rule and tournament solutions
Chapter 7:  Supermajority rules
Chapter 8:  Measuring a-priori voting power (where do I put this?  I am not seriously going to ask them to do math)
Chapter 9:  Condorcet Jury Theorem (goes along with Estlund)
Chapter 14:  Arrow's Theorem
Chapter 15:  Properties and paradoxes of common voting rules
Chapter 17:  multiple-winner rules
The list price for this book is \$225, but it can be had new in hardcover for under \$30.  Nic says he can get a deal for a discount on his book from the current publisher; if the school can secure the book for \$30-ish, I would recommend it as supplemental and not required.  The Handbook of Social Choice and Voting is an excellent text for this course and irreplaceable; Dr. Miller teaches at UMBC and may be able to convince Elgar to supply the book for \$30-\$40 to students via MBS.

I am seriously considering using one week near the end to dig into spatial and probabilistic modeling, and all kinds of complicated analysis used by public choice theorists, specifying that none of this is on any test.  That would be to show the students that there is a future in researching this if they want to do that, but it's a future in mathematics and AI programming at the graduate level.  I have no interest in actually teaching (or understanding) the math, only in making it clear to the students that they've stumbled over a field of study created by the guy who invented calculus and might want to appreciate what that means.

\section{Narrative}

I want this curricula to carry through a certain narrative.  The instructor conveys this by challenging the students in hostile debate:  lead the students to answer challenges from existing knowledge with predictably-wrong answers, then destroy in detail these misconceptions.  This provides two functions in the classroom setting:  first, that the student is heavily invested in the topic and in the ideals which they expect and desire elections to embody, and will be disturbed by threats to these ideals; and second, that the student becomes cautious and guarded, distrusting what they know and putting ever-greater consideration into further challenges.

This works because the student cannot be convinced democracy as they vaguely envision it is an unworthy goal, and rather will be offended by the injustices they never noticed.  The ideal of democracy must be nurtured here, protected, while revealing to the student that they may not be receiving those rights they believe are owed to them.  This is done not by simply telling the student matters of fact, but by drawing out of them a commitment to statements and agreement that these are upheld simply by voting and elections, and then striking at that belief when it is most exposed.

The end goal is to foster a commitment to democracy not as the right to vote, but the right to vote as equals; and to make the student suspicious of anything claiming to be democratic, ready to apply the tools at hand to analyze whether it is in fact democratic, and what must be done to make it more democratic.  New proposals should raise questions about if the current system is in some way deficient, if and how the proposal addresses this, and if it introduces new deficiencies.  Today, many simply do not understand, and can be lead by propaganda, or are automatically suspicious that anything new is an attempt to manipulate the system undemocratically and so act in blind opposition.

To this end, the narrative begins with an attack:  students are encouraged into a discussion about democracy, majority rule, and voting; and then a simple election is used to demonstrate that majority rule is not so simple, that voting is not necessarily democratic, and that what they automatically accepted as fair was quickly shown to be completely broken and undemocratic.

A normative discussion follows, first with Kassner examining if democracy is a process or a set of principles \autocite{Kassner2006}; Condorcet's writings on slavery and the rights of women \autocite{Condorcet1781,Condorcet178}; felony disenfranchisement, black codes, and women's suffrage; and so forth.  Important here is not being able to vote away the rights of others \autocite{Kassner2006,Rawls1997}.  Challenges to policy



In all, this course begins with elections and voting, quickly establishes that majority rule and elections are not so simple, and then moves its way from conceptual ideals to concrete topics.

In the technical side, we examine

The major narrative is outlined below.

\begin{enumerate}

    \item The students, if naive, know nothing about democracy.
    \begin{enumerate}
        \item \technical Immediately undermine the student's fundamental beliefs by raising the usual normative ideal of ``majority rule'' and then showing how various electoral processes fail to achieve anything like this, how government structure affects representativeness, and that the basic questions like ``when is voting democratic?'' and ``what is a majority?'' are complex.

        \item \philosophical Introduce philosophy of democratic legitimacy.  This class targets students who may not be philosophy majors, so will often have little interest in pure philosophy; each approach will come from the backdrop of technical considerations that raise philosophical questions in the student, motivating exploration of those questions.  The first question stems from the above:  is democracy a process or a set of principles?  I have just destroyed the process's ability to adhere to proposed normative principles; the student's full attention is on this question, which is best answered by examining Kassner \autocite{Kassner2006}.  In general, I expect students to converge toward democracy being meaningless if it is purely mechanical, and rather that the mechanical aspects serve to achieve the principles of democracy—otherwise it is absurd to claim democracy exists at all, even as a concept.
    \end{enumerate}

    \item Legitimacy:  what happens to these exercises in collective decisionmaking when not all of the governed are allowed to vote?
    \begin{enumerate}
        \item \philosophical Cite Condorcet's writings on slavery and the political rights of women \autocite{Condorcet1781,Condorcet1789}, where he claims that the properties which give men the right to rule also exist in women, in blacks, and in every person, and so cannot be taken away.

        \item \philosophical Compare this with felony disenfranchisement, and with the Constitution, abolitionist writings, and Supreme Court cases regarding the $\nicefrac{3}{5}$ compromise, slavery in general, black codes, and women's suffrage.

        \item \philosophical In normative terms, these forms of disenfranchisement should be abolished because a democracy involves everyone, not just some people, as established by Condorcet in what exactly conveys rights; by and by both Kassner and Rawls in democracy not legitimately being able to vote away the rights of others \autocite{Kassner2006,Rawls1997}.
    \end{enumerate}

    \item Epistemic democracy:  Consent, utility, ethics, knowledge
    \begin{enumerate}
        \item \technical Some electoral systems focus on preference and consent; others focus on utility.  Condorcet and Smith-efficiency concepts address consent, while Score voting systems address utility and welfare.  The income effect goes here, showing the same amount of dollars may have more value to one person or another, which has huge implications for measuring utility between people.

        \item \philosophical Normative question:  should the election be about what's good for everyone, or what the voters decide by their preferences?  Is it democratic to go by preferences, and ideal but not really democratic to survey what's good for people—even by their own admission—and make the optimal decision for them?

        \item \technical Describe policy ethics frameworks and the distribution of welfare.  Particularly cover Pareto, Kaldor-Hicks, and Scitovsky's Double.  Tideman's book has a good chapter on these \autocite{Tideman2006}, along with the income effect.  Scitovsky's Double is relevant in establishing that a person cannot commit injustice against themselves, and so can vote in favor of having less welfare themselves for others to have more.

        \item \philosophical Cite Estlund on social utility achieved by democracy \autocite{Estlund2008}, and \technical Green-Armytage, Tideman, and Cosman showing the Condorcet winner has high, close to optimal utility \autocite{GreenArmytage2015}.  This opens debate on whether we should use preferential or score-based systems (normative).  Estlund in particular suggests the demos will generally reach the correct conclusions—as noted by Green-Armytage et al, with the Condorcet winner having high if not optimal utility.  This not only justifies democracy, but also preferential democracy.

        \item \philosophical Of particular note, the demos cannot properly achieve utility if it removes voters.  When we make decisions about some people not having the correct knowledge and understanding, the correct beliefs, or otherwise, we are entering a recursive epistemic problem:  if we can improve the likelihood of a correct, optimal outcome by removing people who are wrong, then we must be able to identify who is objectively correct.  This implies the group can continuously vote out voters until the perfect benevolent dictator arises, which would ostensibly provide much greater utility than democracy can achieve—and it would do so through democratically electing the dictator who will then dictate who is the next dictator.  This is internally inconsistent and wholly absurd, and so all forms of disenfranchisement must be absurd and incompatible with the epistemic view of democracy.

        \item \philosophical That's not a normative principle:  \textit{if} the epistemic view is assumed as the normative democratic value, \textit{then} it is positively incorrect and absurd to disenfranchise \textit{any} voter.

        \item \philosophical This raises a bunch of complicated questions about children, who can be shown to be easily influenced and not capable of exercising their own vote; but at the same time, adults are also easily influenced, and are broadly ignorant and unable to exercise their own voting power in any meaningful manner, if we are to assume competence is a prerequisite.  Consider the nature of knowledge:  to know a thing, it must be true; one must believe it is true; and one must have a good reason to believe it is true \autocite{Tideman2006}.

        \item \philosophical Condorcet's views on education, explored by Yastrebtseva \autocite{Yastrebtseva2015}, underscore that education is necessary to avoid tyranny.  At the same time, everyone has the right to vote, which cannot be taken away for lack of education—such as by literacy requirements.  This course covers electoral processes in ways most people don't understand, yet we let those people run for office, make laws, and vote, and they are able to make law about how we vote—this very subject, poorly understood by the voters, and prone to erroneous outcomes.
    \end{enumerate}


    \item Legitimacy:  representative democracy and the public discourse
    \begin{enumerate}
        \item \philosophical Waldron considers judicial review pointless, and claims if there's a question, voters should vote on it \autocite{Waldron1998}.  He discounts the value of Constitutions when we have the process of democracy, and claims judges making judicial review are not compatible with democracy.

        \item \technical Tideman's taxonomy of collective decisions \autocite{Tideman2006} follows Jeremy Waldron's tirade against the Supreme Court because it highlights agreement on outcome versus process.  If we have agreed on certain laws and rules and policies, then why would we have a vote to decide if a new law violates that policy?  We have agreed that someone else shall have the power to do so, and we can have a vote to amend the rules by which they must abide.  We can't usually elect such judges; but we can elect representatives with the power to impeach judges for violating the rules we've set forth.  By the epistemic approach, it's unlikely we'll elect enough representatives to do so unless judges actually misbehave, and they can impeach or otherwise eject one another for violating their own duties.

        \item \philosophical This actually raises a number of questions about getting a government started in the first place.  Governments may rely on some kind of implied consent:  because we all have rights, when we create a government, all must have the voting power necessary to abolish that government.  The force and will of a slim majority of voters is not enough, as a large body of anarchists will in the first place be able to destabilize the government, and in the second can be considered only the oppressed if they are forced to participate in a system where they have a vote \textit{but} it doesn't let them change anything.  We might suggest that governments are based on opinion, as noted by David Hume\footnote{Cite Hume, much as I prefer Kant's ethics.}, and so by Scitovsky's Double it appears that people participating in a democracy not of their own initial design would be the ``losers'' and yet have been convinced to accept the policy of democracy.  This fully resolves itself, from a legitimacy perspective, when the voters begin amending the Constitution and altering the laws, or electing representation to do so, in a way by which they have the opportunity to alter the structure of the government—whether they choose to do so or not—as participating in simply holding the power to modify the government is consent in participating in a government which allows one to do so.

        \item \technical Democracy, as an exercise in collective decisionmaking, is subject to certain technical considerations.  This includes transitive (individual) and intransitive (only possible by the group) preferences.

        \item \philosophical Rawls explores democracy as an exercise of public reason \autocite{Rawls1997}, which raises questions about intransitive group preferences and the income effect.  That a policy is legitimate if it is reasonably believed it will be accepted by all supports the previous suggestions about forming a democratic government.

        \item \technical The two-tier problem? Capacity to focus on so many broad and complex problems?


    \end{enumerate}

    \item Electoral manipulation:  how to screw with the election by voting or nominating a candidate
    \begin{enumerate}
        \item \technical Revisit the opening with plurality, instant runoff voting, and Condorcet.  Show the various systemic failures in terms of weaponization by strategic nomination.

        \item \technical Strategic opportunities tend to be exploited immediately.  Take Borda count in Kiribati \autocite{Reilly2002} as an example.

        \item \technical A segway into intellectual dishonesty is in order here.  Consider score voting as examined by Baujard et al \autocite{Baujard2014}, using an experiment where there is no utility for voting:  the ballots cast have no effect.  Baujard detects little tactical voting and concludes voters really want to vote honestly in real elections, although economists are quite familiar with people's willingness to say they'd do something ethical until faced with the situation and its real costs.

        \item \technical Discuss Feddersen et al \autocite{Feddersen2009}, with much better methodology in which the experiment had real monetary outcomes.  Knowing their vote is likely to affect the outcome, voters voted more in their self-interest; when their vote was likely to not matter, they scored outcomes that were less in their favor higher.  This shows that voters will vote tactically as the chance that their vote changes the outcome increases; and that rather than maximizing the welfare of the group, they seek to maximize their own welfare i.e. self-interest.

        \item \technical Instant runoff, approval, and Score systems all have this flaw wherein a close election can be swung out of the voter's favor if they vote honestly, yet if they vote dishonestly they must give up the opportunity to elect their preferred candidate.  In IRV, voting your favorite can cause your second-favorite to be eliminated when you fall outside the mutual majority, excluding you from the election entirely and producing worse results than even not voting.  With approval, all approved candidates are given a vote against non-approved candidates in the pairwise races, but no vote is cast mutually between approved candidates.  With score, any non-zero score is added to every candidate, acting as a pool of partial votes which may lift a voter's less-preferred candidate from second place to first when their candidate would otherwise win, but withot which may leave a third-choice candidate with more votes than either their first or second choice.

        \item \technical Given that the Condorcet candidate is usually high utility \autocite{GreenArmytage2015} and that score systems appear unviable due to people voting heavily in their self-interest,
    \end{enumerate}

    \item Voting criterion
    \begin{enumerate}
        \item
    \end{enumerate}
\end{enumerate}