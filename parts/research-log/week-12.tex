\section{Week 12}

I rewrote the syllabus section to use a reference counter and custom command, which makes it easier to shuffle things around.

During my research, I came across Quota Borda System (QBS), but it seems poorly-researched and has its own pathologies.  Decided not to pursue this further.  STV-B, using Borda count for eliminations, seems promising.

\subsection{Lessons}

This week I split the voting rules lesson across two lessons, to discuss electoral processes during each.  This allows me to cover party primary cycles top-two elections, and straight runoff during the single-winner lesson; and then cover things like Final Five, STV-Condorcet election cycles, and multi-winner rules in the multi-winner lesson.

In \hyperref[cur:ethics]{Ethics and Welfare}, I expanded on the overview to include discussions of score and Borda voting, and the discussion of voter ethical behavior, candidate and party ethics with the Kiribati case, and Tideman's observation that the Condorcet winner is usually of high welfare and close to the optimal welfare winner.

\hyperref[cur:single-winner-voting-rules]{Single Winner Voting Rules and Elections} now covers what voting rules are, Smith-restricted voting rules, summability, the percieved and actual value of later-no-harm (FairVote leans on this heavily; we talked a lot about FairVote's propaganda at the Public Choice Society's annual member's meeting), and different types of primaries and election cycles.  It also touches on MNTV in Maryland's House of Delegates elections, to compare MNTV to Plurality.

This lesson includes a chart of several voting rules and several criteria they satisfy.

\hyperref[cur:multi-winner-voting-rules]{Multi-Winner Voting Rules and Elections} picks up with SNTV, which our SGA elections and Final Five voting use; MNTV; the serial application of any Condorcet system; Single Transferable Vote; and a brief nod to Quota Borda System, which may be extraneous, but might be worth mentioning to draw attention to the existence of other systems and a lack of exploration (which ties into knowledge and education).

\hyperref[cur:manipulation]{Electoral Manipulation} has a section discussing particular attacks, for the benefit of the instructor.  Instructors should be well-versed with social choice theory and methods of attacking elections.

\subsection{Readings}
I wrote the first draft of the reading on ethics and consent.

