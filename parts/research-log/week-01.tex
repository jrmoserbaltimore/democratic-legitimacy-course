% vim: sw=4 ts=4 et

\section{Week 1}

Assembled a body of social choice theory literature.  Besides the Handbook of Social Choice and Voting \autocite{Heckelman2015} and Nick's book on collective decision making and voting \autocite{Tideman2006}, I collected together various papers I have read over the past years which supply some supporting arguments.

I divided the bibtex database into Social Choice and Philosophy.  Dr. Kassner indicated he would gather some relevant philosophy pieces for me to examine; at the moment, I only have the paper he provided, which argues democracy is a set of principles rather than a mindless process \autocite{Kassner2006}.  This is the core impetus for this course.

I am realizing I will need to examine score voting and the arguments surrounding welfare-based rules.  Personally, I see social choice as a matter of voter preference and not an exercise in maximizing welfare, the latter of which requires many presumptions.  The Wikipedia article on score voting has robust citation of research in this area, and summarizes that people essentially vote more honestly when they think it doesn't matter, and vote strategically if they believe the election will be close.  This suggests people will attempt to sabotage the welfare function if they think it will correctly select the candidate most fitting of society's welfare, assuming said candidate does not maximize their own individual welfare.

Whereas welfare-based systems expect voters to indicate what candidate best suits them but indicate that other candidates are acceptable to some degree, casting a partial vote absolutely for the purpose of conceding that it would be in society's interest—even if it is not the majority opinion—to  focus on what's better for someone else, preferential systems lean more toward voters applying power.  Under welfare rules, a candidate receiving a majority over all other candidates—the Condorcet winner—may often be the one providing the greatest welfare; however it is completely possible that a candidate with no majority support nevertheless tallies up the greatest total welfare score.

My analysis over the years has been that welfare systems are pushed by advocates, and not taken seriously in the field \autocite{AccDemEfficiency}—and for good reason.  Scores range from 1.0 to 0.0; comparing the utility of 1.0 between two voters is ridiculous.  I'm not the only one skeptical of the attempt to compare utility values between different voters; and debates with supporters of STAR voting have raised arguments of the form and depth of ``bayesian regret already accounts for those differences'' when the debate is that it obviously can't.

Merrill finds that the candidate maximizing social welfare is not any more likely to be the Condorcet winner than would be the candidate elected by those non-Condorcet systems studied; but that nevertheless the Condorcet candidate generally has high social utility \autocite{Merrill2014}.  Dr. Kassner mentioned an argument that democracies are most likely to find the ``correct'' or ``best'' answer; assuming democracy as preferential, carrying the logic forward from ``majority rule'' to the phenomena of a Condorcet winner having all majorities, this argument is indistinct from that which claims the Condorcet winner probably has almost as much utility as the candidate with the greatest social utility.  Modern research by Green-Armytage, Tideman, and Cosman indicates when a Condorcet winner exists, rather than a cycle, that candidate is likely the one of highest utility \autocite{GreenArmytage2015}.

These issues of strategic voting, severe flaws in the basic assumptions of voter behavior, assumptions on the comparability of maximum welfare of different voters, lack of consensus support for the viability of any social welfare electoral system, the extreme manipulability of score-based rules as shown by Borda (an interesting case) \autocite{Reilly2002}, and the thin and poorly supported arguments from score voting advocates leads me to believe social welfare maximization systems have no merit.

Not all opinions are equally valid, and I have no intention of portraying damaged systems as having any more merit than is apparent.  Students can make their own decisions on whether they believe that, and the discussion around these systems needs to be complete—incomplete information only leads to discovery of the omitted information later, which frequently moves from ``this would be good if it weren't impossible to actually do'' to ``everything I was taught was wrong and this is obviously totally doable''—but while I'll give fair coverage of all the facts so they can come to their own conclusion, bluntly enumerating and describing the degree to which a broken system is broken \textit{is} fair portrayal when it's actually that broken.  The balance fallacy is a dangerous thing.
