
\section{Week 14}

This is the end of the semester, and this thing really needs a full proofread—which is a long project I'll be engaging in over the summer, because this is far from polished.  Useful and interesting, but not polished.

\subsection{Report}


\subsection{Lessons}

Filled out \hyperref[cur:government-structure]{Government Structure} with ideas I had weeks ago.

\hyperref[cur:representative-democracy]{Representative Democracy} is probably incomplete.

The reading for \hyperref[cur:evaluating-decision-methods]{Evaluating Decision Methods} still needs writing.  The lesson is just one big debate because there's not much to teach during the lesson, and the discussions between material and symbolic equality or about the inefficiency of voting are obvious (especially since this course aims to proliferate knowledge that will lead people to shift our elections from symbolic to material equality by making elections more representative and fair, whether by the means people like Tideman, Green-Armytage, and I have determined or by better means to be discovered by others who possess this knowledge).

\subsection{Research}

I've become somewhat fascinated with STV-B because I can't find a way to break it.  The system is proportional and so manipulation can only go so far, lower rankings have less effect on the Borda score, and there are huge consequences for lowering the ranking of your favored candidates to try and damage other candidates by elevating opponents you think can't win.  There is no safe way to mess with this thing, not without enormous amounts of knowledge about other ballots—the only real defense any system has against sufficiently sophisticated and organized voters.

STV-B is not worse than vote-count elimination, but draws in some of the advantages of Borda.  It always elects the Condorcet candidate as IRV-B with three candidates.  Meek-Geller may partially mitigate some of the cross-over of influence on elimination order, but that's already a trivial problem for many reasons.

That's written into the curricula here, but not necessarily for teaching—at least not without the caviat that the system has been hardly studied and its merits are in reality unknown.  It's there more for the interest of the instructor, who should have good knowledge in social choice theory to begin with.
