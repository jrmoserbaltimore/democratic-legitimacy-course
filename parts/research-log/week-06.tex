
\section{Week 6}

Reading through the handbook of public choice and voting \autocite{Heckelman2015}, note James Buchanan on democratic legitimacy and constitutional unanimous consent in direct contrast with Waldron.

Notably, Buchanan argues that operating under Constitutional rules is Pareto-improving (versus the status quo) even if individual consequences of the rules are not \autocite[37]{Heckelman2015}.  Rawls argues all individuals are in agreement with the social contract if they would be in agreement behind a veil of ignorance, wherein they understand the rules but at the moment of deciding have no idea who they will be in this society \autocite[40]{Heckelman2015} and thus cannot maximize for themselves at the expense of others.

These definitely fit in with the concepts of social contract and welfare ethics in democratic legitimacy:  Pareto is bent somewhat by Buchanan, with a clever and sensible outcome.  Rawls effectively argues that people are ethical when they know the decisions create winners and losers, but don't know if \textit{they} are one of the winners or the losers.
