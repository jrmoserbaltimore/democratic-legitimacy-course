\section{Week 3}

\subsection{Integrating democratic legitimacy}

Question:  how is democratic legitimacy integrated?  As a block here discussing Habermas, Rawls, Rosseau, Waldron, etc., or by pieces throughout?  If throughout, rather than by Phase A/Phase B, how?  I feel integrating the philosophical considerations throughout is more grounded in the goals of this course, being primarily intended to describe how procedure affects elections as a technical consideration, as the sudden shift from philosophy to procedure would simply leave legitimacy behind.

The opposite is also possible:  by the end of this lesson specifically, the foundations of social choice as a concept are lain, and the Process vs. Principles question has been broached; ethics, preference, and welfare are covered as foundations.  Diving directly into attacks on electoral systems, evaluation of their efficiency in representing the group and the individual, Condorcet's jury theorem, and the like is possible; at the same time, the two lessons here raise all kinds of questions when studying the philosophy of democratic legitimacy, as every time someone suggests what \textbf{should} be the goal, we can only reflect on how broken procedures can be, on the difficulty of achieving that goal, and on the ethical implications in the trade of welfare.

In theory, it's possible to integrate a few callbacks into social choice concepts with a philosophical democratic legitimacy discussion, but this seems weaker than bringing the philosophy into a social choice discussion.  The technicals of social choice disrupt philosophical thinking; while philosophical considerations form the basis of normative decisions.  In short: people excited about the philosophy are not as likely to pivot to the procedure; but people engrossed in the implications of procedure for what they term ``democracy'' are likely highly receptive to direct questions and clarifications about what democracy might be and why they are suddenly uncomfortable with the implications of various procedures.