[I need to reread this] Habermas compares theories on what democracy should mean, and points out their flaws.  He indicates citizens do not in practice use political discourse to explore society as a collective identity, as the republican model suggets, as this depends on citizens devoted to the common good.  He presents discourse theory as an explanation connecting liberal and republican theories:  processes of democracy allow people to argue over what normative values they want to adopt as law.  Habermas claims that issues of justice are external to normative ideals, and a law is only legitimate if it is compatible with universal moral tenants (i.e. if it can be said to be just in a reasonable way); that balance cannot be achieved through ethical discourse, due to fundamental conflicts among different parties; by, in the liberal view, fairness granted by equal right to vote, representativeness of parliamentary bodies, and rules of decision making; and otherwise.  Discourse theory considers the Constitution the institutionalization of political norms within which a society agrees to abide in the formation of public political opinion, connecting the liberal concept of a separate state with the republican concept of good-willed ethical discourse.  These structures convert public opinion into influence, then into power through elections, which then are transformed into administrative power by laws.


%
% Kassner's explanation notes:
%Collective decision making = public forum, reasoning leading up to voting. ``Realistic utopia'' practical/politically possible (rawls) - good-faith discussion and deliberation to reach consensus.