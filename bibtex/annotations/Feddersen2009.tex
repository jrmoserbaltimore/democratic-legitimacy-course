Feddersen, Gailmard, and Sandroni examine the proposition that voters will vote ethically so long as this is likely to produce the same results as voting selfishly.  They use an experimental approach with a real payoff, and specifically design their experiment to prevent individuals from ensuring they achieve any benefit by voting ethically rather than by self-interest.  In their experiment, a single active voter is chosen after the round of voting as a dictator, and only their vote counts; they are informed if their vote was pivotal.  As pivot probability increases—as the likelihood of their vote actually determining the outcome increases—voters cast more in self-interest than in concern for the greater social welfare or "ethically."  This is a well-designed experiment and the paper has good methodology.  The implications are important for score and approval voting, suggesting voters will rate candidates below their true utility or approve fewer candidates when the election is perceived to be close; in preferential elections where a voter's further votes are more likely to help than harm them, votes should remain unchanged.
