Authors compare approval and range voting to runoff results using experimental data, focusing on strategic behavior.  The paper has enormous limitations:  for approval and range methods, exit polls at a small number of voting stations (five in total) collected additional experimental ballots from voters.  The authors note a participation bias, as almost half of these voters declined; and that the sample is taken in only a narrow set of stations, not representative of the whole population.  The authors indicate voters more typically preferred to express their honest votes on the experimental ballots, and fail to mention that with zero actual stake in the outcome, voters gain no benefit from strategy, and so would not be expected to vote strategically.  They conclude, instead, that voters voted honestly on these experimental ballots because they truly desire to express themselves honestly, but the official multi-round plurality election essentially forces them to vote strategically.  The omission undermines credibility; and the conclusions drawn appear intellectually dishonest.
