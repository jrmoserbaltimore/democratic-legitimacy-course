Rawls explores democracy as an exercise of public reason, limiting this to judicial decisions and supreme court review of laws; statements and discourse from and between public officials; and political campaigns.  These discourses translate fundamental ideological positions and views of justice into public policy, and into persuasive arguments from public officials and candidates for office attempting to explain why the public policy and political positions they hold follow from these normative concepts of justice.  In this, Rawls claims citizens of a representative democracy have the role of deciding what legislation they believe is ideal and holding elected officials accountable through elections.  Importantly, Rawls indicates public policy is only legitimate if it is reasonably believed that the policy will be accepted by all, a principle he calls reciprocity, and one I find directly in line with my own concept of justice as security—particularly, moral security, in that public policy may prohibit a moral act but may not require an immoral act, as the former restricts behaviors while the latter forces a person to commit what they perceive as evil.
