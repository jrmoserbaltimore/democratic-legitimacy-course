Estlund explores the value of democracy in terms of its social utility derived from the likelihood of a democracy to do the right thing.  This is related to the Condorcet Jury Theorem, in which a group of people making a collective binary decision is more likely to make the right decision if the group is large and each individual is more than 50\% likely to make the right decision, or else is rapidly less likely to make the right decision as the group grows larger.  This piece introduces several considerations and perspectives, without exploration, and serves as an overview of the current state of the art.  Considered alongside Rawls, I might suggest that the basic principle generalizing Jury Theorem implies the discourse Rawls indicated by candidates and legislators justifying their political positions to the citizens illustrates the embodiment of this principle in representative democracy:  a small number of legislators and their staff are able to focus their whole time and effort on evaluating these decisions and their alternatives, and so become much more likely to reach correct solutions than an enormous group of voters with much more limited time on hand to do so.
